\subsection{Software}
We developed a software package with a user interface that implements the above LP problem and allows configuration of clinicians at [ref \ref{???}], to be used by the ID division at St. Michael's Hospital. The software was used to generate the results in the following sections, using real data as well as simulated data as input.

\subsection{Infectious Diseases Division}
We used clinician time-off requests and minimum/maximum requirements from 2015-2018 as input data for the LP problem. Tables \ref{tbl:2017-schedule-comparison}, \ref{tbl:2018-schedule-comparison}, \ref{??}, \ref{??} present the optimal schedule generated using the software as well as the manually created schedule, color-coded to distinguish between the different clinicians assigned.

% Table generated by Excel2LaTeX from sheet '2017'
\begin{table}[h]
	\tiny
	\centering
    \begin{tabular}{c||ccc||ccc}
    	\multicolumn{1}{c||}{\multirow{2}[1]{*}{Week \#}} & \multicolumn{3}{c||}{LP Solution}                                                                                                                                                              & \multicolumn{3}{c}{Historical Data}                                                                                                                                                            \\
    	                                                  &                             HIV                             &                                 ID                                 &                              Weekend                               &                             HIV                             &                                 ID                                 &                              Weekend                               \\ \midrule\midrule
    	                        1                         &            \cellcolor[rgb]{ .608,  .761,  .902}B            &                  \cellcolor[rgb]{ 1,  .851,  .4}C                  &               \cellcolor[rgb]{ .608,  .761,  .902}B                &            \cellcolor[rgb]{ .663,  .816,  .557}A            &               \cellcolor[rgb]{ .788,  .788,  .788}D                &    \cellcolor[rgb]{ .6,  .2,  1}\textcolor[rgb]{ 1,  1,  1}{J}     \\
    	                        2                         &            \cellcolor[rgb]{ .608,  .761,  .902}B            &                  \cellcolor[rgb]{ 1,  .851,  .4}C                  &                \cellcolor[rgb]{ .957,  .69,  .518}E                &            \cellcolor[rgb]{ .663,  .816,  .557}A            &               \cellcolor[rgb]{ .788,  .788,  .788}D                &               \cellcolor[rgb]{ .557,  .663,  .859}F                \\
    	                        3                         &            \cellcolor[rgb]{ .663,  .816,  .557}A            &               \cellcolor[rgb]{ .557,  .663,  .859}F                &               \cellcolor[rgb]{ .557,  .663,  .859}F                &            \cellcolor[rgb]{ .608,  .761,  .902}B            &                \cellcolor[rgb]{ .518,  .592,  .69}G                &               \cellcolor[rgb]{ .663,  .816,  .557}A                \\
    	                        4                         &            \cellcolor[rgb]{ .663,  .816,  .557}A            &               \cellcolor[rgb]{ .557,  .663,  .859}F                &               \cellcolor[rgb]{ .663,  .816,  .557}A                &            \cellcolor[rgb]{ .608,  .761,  .902}B            &                \cellcolor[rgb]{ .518,  .592,  .69}G                &                \cellcolor[rgb]{ .518,  .592,  .69}G                \\
    	                        5                         &            \cellcolor[rgb]{ .608,  .761,  .902}B            &                \cellcolor[rgb]{ .518,  .592,  .69}G                &               \cellcolor[rgb]{ .608,  .761,  .902}B                &            \cellcolor[rgb]{ .663,  .816,  .557}A            &    \cellcolor[rgb]{ .6,  .2,  1}\textcolor[rgb]{ 1,  1,  1}{J}     &                \cellcolor[rgb]{ .957,  .69,  .518}E                \\
    	                        6                         &            \cellcolor[rgb]{ .608,  .761,  .902}B            &                \cellcolor[rgb]{ .518,  .592,  .69}G                & \cellcolor[rgb]{ .251,  .251,  .251}\textcolor[rgb]{ 1,  1,  1}{I} &            \cellcolor[rgb]{ .663,  .816,  .557}A            &    \cellcolor[rgb]{ .6,  .2,  1}\textcolor[rgb]{ 1,  1,  1}{J}     & \cellcolor[rgb]{ .251,  .251,  .251}\textcolor[rgb]{ 1,  1,  1}{I} \\
    	                        7                         & \cellcolor[rgb]{ .6,  .2,  1}\textcolor[rgb]{ 1,  1,  1}{J} &               \cellcolor[rgb]{ .557,  .663,  .859}F                &    \cellcolor[rgb]{ .6,  .2,  1}\textcolor[rgb]{ 1,  1,  1}{J}     &            \cellcolor[rgb]{ .608,  .761,  .902}B            &               \cellcolor[rgb]{ .557,  .663,  .859}F                &                \cellcolor[rgb]{ .957,  .69,  .518}E                \\
    	                        8                         & \cellcolor[rgb]{ .6,  .2,  1}\textcolor[rgb]{ 1,  1,  1}{J} &               \cellcolor[rgb]{ .557,  .663,  .859}F                &                \cellcolor[rgb]{ .518,  .592,  .69}G                &            \cellcolor[rgb]{ .608,  .761,  .902}B            &               \cellcolor[rgb]{ .557,  .663,  .859}F                &                  \cellcolor[rgb]{ 1,  .851,  .4}C                  \\
    	                        9                         &            \cellcolor[rgb]{ .663,  .816,  .557}A            &                \cellcolor[rgb]{ .957,  .69,  .518}E                &               \cellcolor[rgb]{ .663,  .816,  .557}A                & \cellcolor[rgb]{ .6,  .2,  1}\textcolor[rgb]{ 1,  1,  1}{J} &                  \cellcolor[rgb]{ 1,  .851,  .4}C                  &    \cellcolor[rgb]{ .6,  .2,  1}\textcolor[rgb]{ 1,  1,  1}{J}     \\
    	                       10                         &            \cellcolor[rgb]{ .663,  .816,  .557}A            &                \cellcolor[rgb]{ .957,  .69,  .518}E                &                \cellcolor[rgb]{ .957,  .69,  .518}E                &            \cellcolor[rgb]{ .663,  .816,  .557}A            &                  \cellcolor[rgb]{ 1,  .851,  .4}C                  &               \cellcolor[rgb]{ .788,  .788,  .788}D                \\
    	                       11                         &            \cellcolor[rgb]{ .788,  .788,  .788}D            &               \cellcolor[rgb]{ .608,  .761,  .902}B                &               \cellcolor[rgb]{ .788,  .788,  .788}D                &            \cellcolor[rgb]{ .663,  .816,  .557}A            &               \cellcolor[rgb]{ .608,  .761,  .902}B                &               \cellcolor[rgb]{ .663,  .816,  .557}A                \\
    	                       12                         &            \cellcolor[rgb]{ .788,  .788,  .788}D            &               \cellcolor[rgb]{ .608,  .761,  .902}B                &                  \cellcolor[rgb]{ 1,  .851,  .4}C                  & \cellcolor[rgb]{ .6,  .2,  1}\textcolor[rgb]{ 1,  1,  1}{J} & \cellcolor[rgb]{ .251,  .251,  .251}\textcolor[rgb]{ 1,  1,  1}{I} & \cellcolor[rgb]{ .251,  .251,  .251}\textcolor[rgb]{ 1,  1,  1}{I} \\
    	                       13                         & \cellcolor[rgb]{ .6,  .2,  1}\textcolor[rgb]{ 1,  1,  1}{J} & \cellcolor[rgb]{ .251,  .251,  .251}\textcolor[rgb]{ 1,  1,  1}{I} & \cellcolor[rgb]{ .251,  .251,  .251}\textcolor[rgb]{ 1,  1,  1}{I} & \cellcolor[rgb]{ .6,  .2,  1}\textcolor[rgb]{ 1,  1,  1}{J} & \cellcolor[rgb]{ .251,  .251,  .251}\textcolor[rgb]{ 1,  1,  1}{I} &               \cellcolor[rgb]{ .557,  .663,  .859}F                \\
    	                       14                         & \cellcolor[rgb]{ .6,  .2,  1}\textcolor[rgb]{ 1,  1,  1}{J} & \cellcolor[rgb]{ .251,  .251,  .251}\textcolor[rgb]{ 1,  1,  1}{I} &    \cellcolor[rgb]{ .6,  .2,  1}\textcolor[rgb]{ 1,  1,  1}{J}     &            \cellcolor[rgb]{ .608,  .761,  .902}B            &               \cellcolor[rgb]{ .459,  .443,  .443}H                &                \cellcolor[rgb]{ .518,  .592,  .69}G                \\
    	                       15                         &            \cellcolor[rgb]{ .608,  .761,  .902}B            &                \cellcolor[rgb]{ .957,  .69,  .518}E                &                \cellcolor[rgb]{ .957,  .69,  .518}E                &            \cellcolor[rgb]{ .608,  .761,  .902}B            &               \cellcolor[rgb]{ .459,  .443,  .443}H                &               \cellcolor[rgb]{ .459,  .443,  .443}H                \\
    	                       16                         &            \cellcolor[rgb]{ .608,  .761,  .902}B            &                \cellcolor[rgb]{ .957,  .69,  .518}E                &               \cellcolor[rgb]{ .788,  .788,  .788}D                &              \cellcolor[rgb]{ 1,  .851,  .4}C               &                \cellcolor[rgb]{ .957,  .69,  .518}E                &    \cellcolor[rgb]{ .6,  .2,  1}\textcolor[rgb]{ 1,  1,  1}{J}     \\
    	                       17                         &            \cellcolor[rgb]{ .788,  .788,  .788}D            & \cellcolor[rgb]{ .251,  .251,  .251}\textcolor[rgb]{ 1,  1,  1}{I} & \cellcolor[rgb]{ .251,  .251,  .251}\textcolor[rgb]{ 1,  1,  1}{I} &              \cellcolor[rgb]{ 1,  .851,  .4}C               &                \cellcolor[rgb]{ .957,  .69,  .518}E                &                \cellcolor[rgb]{ .957,  .69,  .518}E                \\
    	                       18                         &            \cellcolor[rgb]{ .788,  .788,  .788}D            & \cellcolor[rgb]{ .251,  .251,  .251}\textcolor[rgb]{ 1,  1,  1}{I} &               \cellcolor[rgb]{ .557,  .663,  .859}F                &            \cellcolor[rgb]{ .663,  .816,  .557}A            &                \cellcolor[rgb]{ .518,  .592,  .69}G                &               \cellcolor[rgb]{ .557,  .663,  .859}F                \\
    	                       19                         &              \cellcolor[rgb]{ 1,  .851,  .4}C               &               \cellcolor[rgb]{ .459,  .443,  .443}H                &                  \cellcolor[rgb]{ 1,  .851,  .4}C                  &            \cellcolor[rgb]{ .663,  .816,  .557}A            &                \cellcolor[rgb]{ .518,  .592,  .69}G                &                \cellcolor[rgb]{ .518,  .592,  .69}G                \\
    	                       20                         &              \cellcolor[rgb]{ 1,  .851,  .4}C               &               \cellcolor[rgb]{ .459,  .443,  .443}H                &                \cellcolor[rgb]{ .518,  .592,  .69}G                & \cellcolor[rgb]{ .6,  .2,  1}\textcolor[rgb]{ 1,  1,  1}{J} &               \cellcolor[rgb]{ .557,  .663,  .859}F                &               \cellcolor[rgb]{ .663,  .816,  .557}A                \\
    	                       21                         &            \cellcolor[rgb]{ .663,  .816,  .557}A            &                \cellcolor[rgb]{ .957,  .69,  .518}E                &                \cellcolor[rgb]{ .957,  .69,  .518}E                &            \cellcolor[rgb]{ .788,  .788,  .788}D            &               \cellcolor[rgb]{ .557,  .663,  .859}F                &                  \cellcolor[rgb]{ 1,  .851,  .4}C                  \\
    	                       22                         &            \cellcolor[rgb]{ .663,  .816,  .557}A            &                \cellcolor[rgb]{ .957,  .69,  .518}E                & \cellcolor[rgb]{ .251,  .251,  .251}\textcolor[rgb]{ 1,  1,  1}{I} &            \cellcolor[rgb]{ .788,  .788,  .788}D            &                  \cellcolor[rgb]{ 1,  .851,  .4}C                  & \cellcolor[rgb]{ .251,  .251,  .251}\textcolor[rgb]{ 1,  1,  1}{I} \\
    	                       23                         &            \cellcolor[rgb]{ .788,  .788,  .788}D            &               \cellcolor[rgb]{ .608,  .761,  .902}B                &               \cellcolor[rgb]{ .608,  .761,  .902}B                &            \cellcolor[rgb]{ .608,  .761,  .902}B            &                  \cellcolor[rgb]{ 1,  .851,  .4}C                  &               \cellcolor[rgb]{ .459,  .443,  .443}H                \\
    	                       24                         &            \cellcolor[rgb]{ .788,  .788,  .788}D            &               \cellcolor[rgb]{ .608,  .761,  .902}B                &               \cellcolor[rgb]{ .459,  .443,  .443}H                &            \cellcolor[rgb]{ .608,  .761,  .902}B            &                \cellcolor[rgb]{ .957,  .69,  .518}E                &               \cellcolor[rgb]{ .788,  .788,  .788}D                \\
    	                       25                         &            \cellcolor[rgb]{ .663,  .816,  .557}A            &                  \cellcolor[rgb]{ 1,  .851,  .4}C                  &               \cellcolor[rgb]{ .663,  .816,  .557}A                &            \cellcolor[rgb]{ .608,  .761,  .902}B            &                \cellcolor[rgb]{ .957,  .69,  .518}E                &               \cellcolor[rgb]{ .663,  .816,  .557}A                \\
    	                       26                         &            \cellcolor[rgb]{ .663,  .816,  .557}A            &                  \cellcolor[rgb]{ 1,  .851,  .4}C                  &    \cellcolor[rgb]{ .6,  .2,  1}\textcolor[rgb]{ 1,  1,  1}{J}     &            \cellcolor[rgb]{ .788,  .788,  .788}D            &               \cellcolor[rgb]{ .557,  .663,  .859}F                &               \cellcolor[rgb]{ .557,  .663,  .859}F                \\
    	                       27                         & \cellcolor[rgb]{ .6,  .2,  1}\textcolor[rgb]{ 1,  1,  1}{J} &               \cellcolor[rgb]{ .788,  .788,  .788}D                &               \cellcolor[rgb]{ .788,  .788,  .788}D                &            \cellcolor[rgb]{ .788,  .788,  .788}D            &               \cellcolor[rgb]{ .557,  .663,  .859}F                &                  \cellcolor[rgb]{ 1,  .851,  .4}C                  \\
    	                       28                         & \cellcolor[rgb]{ .6,  .2,  1}\textcolor[rgb]{ 1,  1,  1}{J} &               \cellcolor[rgb]{ .788,  .788,  .788}D                &               \cellcolor[rgb]{ .459,  .443,  .443}H                &            \cellcolor[rgb]{ .663,  .816,  .557}A            &                  \cellcolor[rgb]{ 1,  .851,  .4}C                  &               \cellcolor[rgb]{ .788,  .788,  .788}D                \\
    	                       29                         &            \cellcolor[rgb]{ .608,  .761,  .902}B            &               \cellcolor[rgb]{ .459,  .443,  .443}H                &               \cellcolor[rgb]{ .608,  .761,  .902}B                &            \cellcolor[rgb]{ .663,  .816,  .557}A            &                  \cellcolor[rgb]{ 1,  .851,  .4}C                  &               \cellcolor[rgb]{ .459,  .443,  .443}H                \\
    	                       30                         &            \cellcolor[rgb]{ .608,  .761,  .902}B            &               \cellcolor[rgb]{ .459,  .443,  .443}H                &               \cellcolor[rgb]{ .557,  .663,  .859}F                &            \cellcolor[rgb]{ .608,  .761,  .902}B            &               \cellcolor[rgb]{ .459,  .443,  .443}H                & \cellcolor[rgb]{ .251,  .251,  .251}\textcolor[rgb]{ 1,  1,  1}{I} \\
    	                       31                         &            \cellcolor[rgb]{ .663,  .816,  .557}A            &               \cellcolor[rgb]{ .557,  .663,  .859}F                &               \cellcolor[rgb]{ .663,  .816,  .557}A                &            \cellcolor[rgb]{ .608,  .761,  .902}B            &               \cellcolor[rgb]{ .459,  .443,  .443}H                &               \cellcolor[rgb]{ .608,  .761,  .902}B                \\
    	                       32                         &            \cellcolor[rgb]{ .663,  .816,  .557}A            &               \cellcolor[rgb]{ .557,  .663,  .859}F                &    \cellcolor[rgb]{ .6,  .2,  1}\textcolor[rgb]{ 1,  1,  1}{J}     &            \cellcolor[rgb]{ .663,  .816,  .557}A            &    \cellcolor[rgb]{ .6,  .2,  1}\textcolor[rgb]{ 1,  1,  1}{J}     &                \cellcolor[rgb]{ .957,  .69,  .518}E                \\
    	                       33                         &            \cellcolor[rgb]{ .608,  .761,  .902}B            &                  \cellcolor[rgb]{ 1,  .851,  .4}C                  &                  \cellcolor[rgb]{ 1,  .851,  .4}C                  &            \cellcolor[rgb]{ .608,  .761,  .902}B            &                \cellcolor[rgb]{ .957,  .69,  .518}E                &               \cellcolor[rgb]{ .788,  .788,  .788}D                \\
    	                       34                         &            \cellcolor[rgb]{ .608,  .761,  .902}B            &                  \cellcolor[rgb]{ 1,  .851,  .4}C                  &                \cellcolor[rgb]{ .957,  .69,  .518}E                &            \cellcolor[rgb]{ .608,  .761,  .902}B            &                \cellcolor[rgb]{ .957,  .69,  .518}E                &               \cellcolor[rgb]{ .663,  .816,  .557}A                \\
    	                       35                         &            \cellcolor[rgb]{ .663,  .816,  .557}A            &               \cellcolor[rgb]{ .557,  .663,  .859}F                &               \cellcolor[rgb]{ .557,  .663,  .859}F                &            \cellcolor[rgb]{ .663,  .816,  .557}A            &                \cellcolor[rgb]{ .518,  .592,  .69}G                &    \cellcolor[rgb]{ .6,  .2,  1}\textcolor[rgb]{ 1,  1,  1}{J}     \\
    	                       36                         &            \cellcolor[rgb]{ .663,  .816,  .557}A            &               \cellcolor[rgb]{ .557,  .663,  .859}F                &                \cellcolor[rgb]{ .518,  .592,  .69}G                &            \cellcolor[rgb]{ .663,  .816,  .557}A            &    \cellcolor[rgb]{ .6,  .2,  1}\textcolor[rgb]{ 1,  1,  1}{J}     &               \cellcolor[rgb]{ .663,  .816,  .557}A                \\
    	                       37                         & \cellcolor[rgb]{ .6,  .2,  1}\textcolor[rgb]{ 1,  1,  1}{J} &                \cellcolor[rgb]{ .957,  .69,  .518}E                &    \cellcolor[rgb]{ .6,  .2,  1}\textcolor[rgb]{ 1,  1,  1}{J}     &            \cellcolor[rgb]{ .663,  .816,  .557}A            &    \cellcolor[rgb]{ .6,  .2,  1}\textcolor[rgb]{ 1,  1,  1}{J}     &                \cellcolor[rgb]{ .518,  .592,  .69}G                \\
    	                       38                         & \cellcolor[rgb]{ .6,  .2,  1}\textcolor[rgb]{ 1,  1,  1}{J} &                \cellcolor[rgb]{ .957,  .69,  .518}E                &               \cellcolor[rgb]{ .663,  .816,  .557}A                &            \cellcolor[rgb]{ .788,  .788,  .788}D            &               \cellcolor[rgb]{ .608,  .761,  .902}B                &    \cellcolor[rgb]{ .6,  .2,  1}\textcolor[rgb]{ 1,  1,  1}{J}     \\
    	                       39                         &              \cellcolor[rgb]{ 1,  .851,  .4}C               & \cellcolor[rgb]{ .251,  .251,  .251}\textcolor[rgb]{ 1,  1,  1}{I} &                  \cellcolor[rgb]{ 1,  .851,  .4}C                  &            \cellcolor[rgb]{ .788,  .788,  .788}D            &               \cellcolor[rgb]{ .459,  .443,  .443}H                & \cellcolor[rgb]{ .251,  .251,  .251}\textcolor[rgb]{ 1,  1,  1}{I} \\
    	                       40                         &              \cellcolor[rgb]{ 1,  .851,  .4}C               & \cellcolor[rgb]{ .251,  .251,  .251}\textcolor[rgb]{ 1,  1,  1}{I} &               \cellcolor[rgb]{ .459,  .443,  .443}H                &              \cellcolor[rgb]{ 1,  .851,  .4}C               &               \cellcolor[rgb]{ .459,  .443,  .443}H                &               \cellcolor[rgb]{ .788,  .788,  .788}D                \\
    	                       41                         &            \cellcolor[rgb]{ .608,  .761,  .902}B            &                \cellcolor[rgb]{ .518,  .592,  .69}G                &                \cellcolor[rgb]{ .518,  .592,  .69}G                &              \cellcolor[rgb]{ 1,  .851,  .4}C               &               \cellcolor[rgb]{ .608,  .761,  .902}B                &                  \cellcolor[rgb]{ 1,  .851,  .4}C                  \\
    	                       42                         &            \cellcolor[rgb]{ .608,  .761,  .902}B            &                \cellcolor[rgb]{ .518,  .592,  .69}G                &                \cellcolor[rgb]{ .957,  .69,  .518}E                &            \cellcolor[rgb]{ .608,  .761,  .902}B            &               \cellcolor[rgb]{ .557,  .663,  .859}F                &               \cellcolor[rgb]{ .557,  .663,  .859}F                \\
    	                       43                         &              \cellcolor[rgb]{ 1,  .851,  .4}C               &               \cellcolor[rgb]{ .788,  .788,  .788}D                &               \cellcolor[rgb]{ .788,  .788,  .788}D                &            \cellcolor[rgb]{ .608,  .761,  .902}B            &               \cellcolor[rgb]{ .557,  .663,  .859}F                &               \cellcolor[rgb]{ .788,  .788,  .788}D                \\
    	                       44                         &              \cellcolor[rgb]{ 1,  .851,  .4}C               &               \cellcolor[rgb]{ .788,  .788,  .788}D                & \cellcolor[rgb]{ .251,  .251,  .251}\textcolor[rgb]{ 1,  1,  1}{I} &            \cellcolor[rgb]{ .663,  .816,  .557}A            &    \cellcolor[rgb]{ .6,  .2,  1}\textcolor[rgb]{ 1,  1,  1}{J}     &               \cellcolor[rgb]{ .459,  .443,  .443}H                \\
    	                       45                         &            \cellcolor[rgb]{ .663,  .816,  .557}A            &               \cellcolor[rgb]{ .459,  .443,  .443}H                &               \cellcolor[rgb]{ .459,  .443,  .443}H                &            \cellcolor[rgb]{ .663,  .816,  .557}A            &    \cellcolor[rgb]{ .6,  .2,  1}\textcolor[rgb]{ 1,  1,  1}{J}     &                  \cellcolor[rgb]{ 1,  .851,  .4}C                  \\
    	                       46                         &            \cellcolor[rgb]{ .663,  .816,  .557}A            &               \cellcolor[rgb]{ .459,  .443,  .443}H                &               \cellcolor[rgb]{ .608,  .761,  .902}B                &            \cellcolor[rgb]{ .608,  .761,  .902}B            & \cellcolor[rgb]{ .251,  .251,  .251}\textcolor[rgb]{ 1,  1,  1}{I} &                \cellcolor[rgb]{ .518,  .592,  .69}G                \\
    	                       47                         &            \cellcolor[rgb]{ .608,  .761,  .902}B            &               \cellcolor[rgb]{ .788,  .788,  .788}D                &               \cellcolor[rgb]{ .788,  .788,  .788}D                &            \cellcolor[rgb]{ .608,  .761,  .902}B            &               \cellcolor[rgb]{ .557,  .663,  .859}F                &                \cellcolor[rgb]{ .957,  .69,  .518}E                \\
    	                       48                         &            \cellcolor[rgb]{ .608,  .761,  .902}B            &               \cellcolor[rgb]{ .788,  .788,  .788}D                &                  \cellcolor[rgb]{ 1,  .851,  .4}C                  &            \cellcolor[rgb]{ .788,  .788,  .788}D            &               \cellcolor[rgb]{ .608,  .761,  .902}B                & \cellcolor[rgb]{ .251,  .251,  .251}\textcolor[rgb]{ 1,  1,  1}{I} \\
    	                       49                         &            \cellcolor[rgb]{ .663,  .816,  .557}A            &               \cellcolor[rgb]{ .459,  .443,  .443}H                &               \cellcolor[rgb]{ .459,  .443,  .443}H                &            \cellcolor[rgb]{ .663,  .816,  .557}A            &               \cellcolor[rgb]{ .608,  .761,  .902}B                &               \cellcolor[rgb]{ .459,  .443,  .443}H                \\
    	                       50                         &            \cellcolor[rgb]{ .663,  .816,  .557}A            &               \cellcolor[rgb]{ .459,  .443,  .443}H                &               \cellcolor[rgb]{ .608,  .761,  .902}B                &              \cellcolor[rgb]{ 1,  .851,  .4}C               &               \cellcolor[rgb]{ .788,  .788,  .788}D                &               \cellcolor[rgb]{ .557,  .663,  .859}F                \\
    	                       51                         &            \cellcolor[rgb]{ .608,  .761,  .902}B            &                \cellcolor[rgb]{ .518,  .592,  .69}G                &                \cellcolor[rgb]{ .518,  .592,  .69}G                &              \cellcolor[rgb]{ 1,  .851,  .4}C               &               \cellcolor[rgb]{ .788,  .788,  .788}D                &                \cellcolor[rgb]{ .957,  .69,  .518}E
    \end{tabular}%
	\caption{Comparison of schedules for 2017}
	\label{tbl:2017-schedule-comparison}%
\end{table}%

% Table generated by Excel2LaTeX from sheet '2018'
\begin{table}[h]
%	\tiny
 	\centering
 	\begin{adjustbox}{scale=0.8}
	    \begin{tabular}{c||ccc||ccc}
	    	\multicolumn{1}{c||}{\multirow{2}[1]{*}{Week \#}} & \multicolumn{3}{c||}{LP Solution}                                                                                                                                       & \multicolumn{3}{c}{Historical Data}                                                                                                                                     \\
	    	                                                  &                  HIV                  &                                 ID                                 &                              Weekend                               &                  HIV                  &                                 ID                                 &                              Weekend                               \\ \midrule\midrule
	    	                        1                         & \cellcolor[rgb]{ .663,  .816,  .557}A &                \cellcolor[rgb]{ .957,  .69,  .518}E                &                \cellcolor[rgb]{ .957,  .69,  .518}E                & \cellcolor[rgb]{ .663,  .816,  .557}A &                \cellcolor[rgb]{ .957,  .69,  .518}E                &               \cellcolor[rgb]{ .459,  .443,  .443}H                \\
	    	                        2                         & \cellcolor[rgb]{ .663,  .816,  .557}A &                \cellcolor[rgb]{ .957,  .69,  .518}E                &                \cellcolor[rgb]{ .518,  .592,  .69}G                & \cellcolor[rgb]{ .663,  .816,  .557}A &                \cellcolor[rgb]{ .957,  .69,  .518}E                &               \cellcolor[rgb]{ .663,  .816,  .557}A                \\
	    	                        3                         & \cellcolor[rgb]{ .608,  .761,  .902}B &               \cellcolor[rgb]{ .557,  .663,  .859}F                &               \cellcolor[rgb]{ .557,  .663,  .859}F                & \cellcolor[rgb]{ .608,  .761,  .902}B &               \cellcolor[rgb]{ .459,  .443,  .443}H                &                \cellcolor[rgb]{ .518,  .592,  .69}G                \\
	    	                        4                         & \cellcolor[rgb]{ .608,  .761,  .902}B &               \cellcolor[rgb]{ .557,  .663,  .859}F                &               \cellcolor[rgb]{ .459,  .443,  .443}H                & \cellcolor[rgb]{ .608,  .761,  .902}B &               \cellcolor[rgb]{ .459,  .443,  .443}H                & \cellcolor[rgb]{ .251,  .251,  .251}\textcolor[rgb]{ 1,  1,  1}{I} \\
	    	                        5                         & \cellcolor[rgb]{ .663,  .816,  .557}A &                \cellcolor[rgb]{ .518,  .592,  .69}G                &               \cellcolor[rgb]{ .663,  .816,  .557}A                & \cellcolor[rgb]{ .663,  .816,  .557}A &                \cellcolor[rgb]{ .518,  .592,  .69}G                &               \cellcolor[rgb]{ .557,  .663,  .859}F                \\
	    	                        6                         & \cellcolor[rgb]{ .663,  .816,  .557}A &                \cellcolor[rgb]{ .518,  .592,  .69}G                &                \cellcolor[rgb]{ .957,  .69,  .518}E                & \cellcolor[rgb]{ .663,  .816,  .557}A &                \cellcolor[rgb]{ .518,  .592,  .69}G                &                  \cellcolor[rgb]{ 1,  .851,  .4}C                  \\
	    	                        7                         &   \cellcolor[rgb]{ 1,  .851,  .4}C    &               \cellcolor[rgb]{ .608,  .761,  .902}B                &                  \cellcolor[rgb]{ 1,  .851,  .4}C                  & \cellcolor[rgb]{ .663,  .816,  .557}A &               \cellcolor[rgb]{ .557,  .663,  .859}F                &               \cellcolor[rgb]{ .608,  .761,  .902}B                \\
	    	                        8                         &   \cellcolor[rgb]{ 1,  .851,  .4}C    &               \cellcolor[rgb]{ .608,  .761,  .902}B                &                \cellcolor[rgb]{ .518,  .592,  .69}G                & \cellcolor[rgb]{ .788,  .788,  .788}D &                  \cellcolor[rgb]{ 1,  .851,  .4}C                  &                \cellcolor[rgb]{ .518,  .592,  .69}G                \\
	    	                        9                         & \cellcolor[rgb]{ .788,  .788,  .788}D &               \cellcolor[rgb]{ .459,  .443,  .443}H                &               \cellcolor[rgb]{ .788,  .788,  .788}D                & \cellcolor[rgb]{ .608,  .761,  .902}B &                  \cellcolor[rgb]{ 1,  .851,  .4}C                  &               \cellcolor[rgb]{ .788,  .788,  .788}D                \\
	    	                       10                         & \cellcolor[rgb]{ .788,  .788,  .788}D &               \cellcolor[rgb]{ .459,  .443,  .443}H                &               \cellcolor[rgb]{ .459,  .443,  .443}H                & \cellcolor[rgb]{ .608,  .761,  .902}B &               \cellcolor[rgb]{ .788,  .788,  .788}D                &               \cellcolor[rgb]{ .459,  .443,  .443}H                \\
	    	                       11                         & \cellcolor[rgb]{ .663,  .816,  .557}A & \cellcolor[rgb]{ .251,  .251,  .251}\textcolor[rgb]{ 1,  1,  1}{I} & \cellcolor[rgb]{ .251,  .251,  .251}\textcolor[rgb]{ 1,  1,  1}{I} & \cellcolor[rgb]{ .663,  .816,  .557}A &               \cellcolor[rgb]{ .608,  .761,  .902}B                &               \cellcolor[rgb]{ .557,  .663,  .859}F                \\
	    	                       12                         & \cellcolor[rgb]{ .663,  .816,  .557}A & \cellcolor[rgb]{ .251,  .251,  .251}\textcolor[rgb]{ 1,  1,  1}{I} &               \cellcolor[rgb]{ .608,  .761,  .902}B                & \cellcolor[rgb]{ .663,  .816,  .557}A &               \cellcolor[rgb]{ .608,  .761,  .902}B                &               \cellcolor[rgb]{ .663,  .816,  .557}A                \\
	    	                       13                         & \cellcolor[rgb]{ .608,  .761,  .902}B &               \cellcolor[rgb]{ .557,  .663,  .859}F                &               \cellcolor[rgb]{ .557,  .663,  .859}F                &   \cellcolor[rgb]{ 1,  .851,  .4}C    &               \cellcolor[rgb]{ .459,  .443,  .443}H                &               \cellcolor[rgb]{ .459,  .443,  .443}H                \\
	    	                       14                         & \cellcolor[rgb]{ .608,  .761,  .902}B &               \cellcolor[rgb]{ .557,  .663,  .859}F                &               \cellcolor[rgb]{ .459,  .443,  .443}H                &   \cellcolor[rgb]{ 1,  .851,  .4}C    &               \cellcolor[rgb]{ .459,  .443,  .443}H                & \cellcolor[rgb]{ .251,  .251,  .251}\textcolor[rgb]{ 1,  1,  1}{I} \\
	    	                       15                         &   \cellcolor[rgb]{ 1,  .851,  .4}C    & \cellcolor[rgb]{ .251,  .251,  .251}\textcolor[rgb]{ 1,  1,  1}{I} &                  \cellcolor[rgb]{ 1,  .851,  .4}C                  & \cellcolor[rgb]{ .608,  .761,  .902}B & \cellcolor[rgb]{ .251,  .251,  .251}\textcolor[rgb]{ 1,  1,  1}{I} &                  \cellcolor[rgb]{ 1,  .851,  .4}C                  \\
	    	                       16                         &   \cellcolor[rgb]{ 1,  .851,  .4}C    & \cellcolor[rgb]{ .251,  .251,  .251}\textcolor[rgb]{ 1,  1,  1}{I} &               \cellcolor[rgb]{ .663,  .816,  .557}A                & \cellcolor[rgb]{ .608,  .761,  .902}B & \cellcolor[rgb]{ .251,  .251,  .251}\textcolor[rgb]{ 1,  1,  1}{I} &                \cellcolor[rgb]{ .957,  .69,  .518}E                \\
	    	                       17                         & \cellcolor[rgb]{ .608,  .761,  .902}B &               \cellcolor[rgb]{ .788,  .788,  .788}D                &               \cellcolor[rgb]{ .788,  .788,  .788}D                & \cellcolor[rgb]{ .663,  .816,  .557}A &                \cellcolor[rgb]{ .957,  .69,  .518}E                &               \cellcolor[rgb]{ .788,  .788,  .788}D                \\
	    	                       18                         & \cellcolor[rgb]{ .608,  .761,  .902}B &               \cellcolor[rgb]{ .788,  .788,  .788}D                &               \cellcolor[rgb]{ .557,  .663,  .859}F                & \cellcolor[rgb]{ .663,  .816,  .557}A &                \cellcolor[rgb]{ .957,  .69,  .518}E                &                \cellcolor[rgb]{ .957,  .69,  .518}E                \\
	    	                       19                         & \cellcolor[rgb]{ .663,  .816,  .557}A &               \cellcolor[rgb]{ .459,  .443,  .443}H                &               \cellcolor[rgb]{ .459,  .443,  .443}H                & \cellcolor[rgb]{ .663,  .816,  .557}A &                  \cellcolor[rgb]{ 1,  .851,  .4}C                  &               \cellcolor[rgb]{ .557,  .663,  .859}F                \\
	    	                       20                         & \cellcolor[rgb]{ .663,  .816,  .557}A &               \cellcolor[rgb]{ .459,  .443,  .443}H                & \cellcolor[rgb]{ .251,  .251,  .251}\textcolor[rgb]{ 1,  1,  1}{I} & \cellcolor[rgb]{ .663,  .816,  .557}A &                  \cellcolor[rgb]{ 1,  .851,  .4}C                  &                  \cellcolor[rgb]{ 1,  .851,  .4}C                  \\
	    	                       21                         & \cellcolor[rgb]{ .788,  .788,  .788}D &                  \cellcolor[rgb]{ 1,  .851,  .4}C                  &                  \cellcolor[rgb]{ 1,  .851,  .4}C                  & \cellcolor[rgb]{ .608,  .761,  .902}B &                \cellcolor[rgb]{ .518,  .592,  .69}G                &               \cellcolor[rgb]{ .663,  .816,  .557}A                \\
	    	                       22                         & \cellcolor[rgb]{ .788,  .788,  .788}D &                  \cellcolor[rgb]{ 1,  .851,  .4}C                  &                \cellcolor[rgb]{ .518,  .592,  .69}G                & \cellcolor[rgb]{ .608,  .761,  .902}B &                \cellcolor[rgb]{ .518,  .592,  .69}G                &                  \cellcolor[rgb]{ 1,  .851,  .4}C                  \\
	    	                       23                         & \cellcolor[rgb]{ .608,  .761,  .902}B &                \cellcolor[rgb]{ .957,  .69,  .518}E                &                \cellcolor[rgb]{ .957,  .69,  .518}E                &   \cellcolor[rgb]{ 1,  .851,  .4}C    &               \cellcolor[rgb]{ .557,  .663,  .859}F                &               \cellcolor[rgb]{ .788,  .788,  .788}D                \\
	    	                       24                         & \cellcolor[rgb]{ .608,  .761,  .902}B &                \cellcolor[rgb]{ .957,  .69,  .518}E                &               \cellcolor[rgb]{ .557,  .663,  .859}F                &   \cellcolor[rgb]{ 1,  .851,  .4}C    &               \cellcolor[rgb]{ .557,  .663,  .859}F                &                  \cellcolor[rgb]{ 1,  .851,  .4}C                  \\
	    	                       25                         & \cellcolor[rgb]{ .663,  .816,  .557}A &               \cellcolor[rgb]{ .459,  .443,  .443}H                &               \cellcolor[rgb]{ .663,  .816,  .557}A                &   \cellcolor[rgb]{ 1,  .851,  .4}C    &                  \cellcolor[rgb]{ 1,  .851,  .4}C                  &                \cellcolor[rgb]{ .518,  .592,  .69}G                \\
	    	                       26                         & \cellcolor[rgb]{ .663,  .816,  .557}A &               \cellcolor[rgb]{ .459,  .443,  .443}H                &               \cellcolor[rgb]{ .459,  .443,  .443}H                & \cellcolor[rgb]{ .788,  .788,  .788}D & \cellcolor[rgb]{ .251,  .251,  .251}\textcolor[rgb]{ 1,  1,  1}{I} &               \cellcolor[rgb]{ .788,  .788,  .788}D                \\
	    	                       27                         & \cellcolor[rgb]{ .608,  .761,  .902}B &                  \cellcolor[rgb]{ 1,  .851,  .4}C                  &                  \cellcolor[rgb]{ 1,  .851,  .4}C                  & \cellcolor[rgb]{ .663,  .816,  .557}A &               \cellcolor[rgb]{ .608,  .761,  .902}B                &                \cellcolor[rgb]{ .957,  .69,  .518}E                \\
	    	                       28                         & \cellcolor[rgb]{ .608,  .761,  .902}B &                  \cellcolor[rgb]{ 1,  .851,  .4}C                  &                \cellcolor[rgb]{ .957,  .69,  .518}E                & \cellcolor[rgb]{ .663,  .816,  .557}A &               \cellcolor[rgb]{ .608,  .761,  .902}B                & \cellcolor[rgb]{ .251,  .251,  .251}\textcolor[rgb]{ 1,  1,  1}{I} \\
	    	                       29                         & \cellcolor[rgb]{ .663,  .816,  .557}A &                \cellcolor[rgb]{ .518,  .592,  .69}G                &               \cellcolor[rgb]{ .663,  .816,  .557}A                & \cellcolor[rgb]{ .608,  .761,  .902}B &               \cellcolor[rgb]{ .788,  .788,  .788}D                &               \cellcolor[rgb]{ .788,  .788,  .788}D                \\
	    	                       30                         & \cellcolor[rgb]{ .663,  .816,  .557}A &                \cellcolor[rgb]{ .518,  .592,  .69}G                &               \cellcolor[rgb]{ .608,  .761,  .902}B                & \cellcolor[rgb]{ .608,  .761,  .902}B &               \cellcolor[rgb]{ .788,  .788,  .788}D                &               \cellcolor[rgb]{ .663,  .816,  .557}A                \\
	    	                       31                         &   \cellcolor[rgb]{ 1,  .851,  .4}C    &               \cellcolor[rgb]{ .557,  .663,  .859}F                &                  \cellcolor[rgb]{ 1,  .851,  .4}C                  &   \cellcolor[rgb]{ 1,  .851,  .4}C    &               \cellcolor[rgb]{ .557,  .663,  .859}F                &                \cellcolor[rgb]{ .957,  .69,  .518}E                \\
	    	                       32                         &   \cellcolor[rgb]{ 1,  .851,  .4}C    &               \cellcolor[rgb]{ .557,  .663,  .859}F                &                \cellcolor[rgb]{ .957,  .69,  .518}E                &   \cellcolor[rgb]{ 1,  .851,  .4}C    &               \cellcolor[rgb]{ .557,  .663,  .859}F                &               \cellcolor[rgb]{ .557,  .663,  .859}F                \\
	    	                       33                         & \cellcolor[rgb]{ .608,  .761,  .902}B &               \cellcolor[rgb]{ .788,  .788,  .788}D                &               \cellcolor[rgb]{ .788,  .788,  .788}D                & \cellcolor[rgb]{ .608,  .761,  .902}B &               \cellcolor[rgb]{ .557,  .663,  .859}F                & \cellcolor[rgb]{ .251,  .251,  .251}\textcolor[rgb]{ 1,  1,  1}{I} \\
	    	                       34                         & \cellcolor[rgb]{ .608,  .761,  .902}B &               \cellcolor[rgb]{ .788,  .788,  .788}D                &               \cellcolor[rgb]{ .608,  .761,  .902}B                & \cellcolor[rgb]{ .608,  .761,  .902}B & \cellcolor[rgb]{ .251,  .251,  .251}\textcolor[rgb]{ 1,  1,  1}{I} &                  \cellcolor[rgb]{ 1,  .851,  .4}C                  \\
	    	                       35                         & \cellcolor[rgb]{ .663,  .816,  .557}A & \cellcolor[rgb]{ .251,  .251,  .251}\textcolor[rgb]{ 1,  1,  1}{I} & \cellcolor[rgb]{ .251,  .251,  .251}\textcolor[rgb]{ 1,  1,  1}{I} & \cellcolor[rgb]{ .663,  .816,  .557}A &                \cellcolor[rgb]{ .518,  .592,  .69}G                &                \cellcolor[rgb]{ .518,  .592,  .69}G                \\
	    	                       36                         & \cellcolor[rgb]{ .663,  .816,  .557}A & \cellcolor[rgb]{ .251,  .251,  .251}\textcolor[rgb]{ 1,  1,  1}{I} &                \cellcolor[rgb]{ .518,  .592,  .69}G                & \cellcolor[rgb]{ .663,  .816,  .557}A &                \cellcolor[rgb]{ .518,  .592,  .69}G                & \cellcolor[rgb]{ .251,  .251,  .251}\textcolor[rgb]{ 1,  1,  1}{I} \\
	    	                       37                         & \cellcolor[rgb]{ .788,  .788,  .788}D &                \cellcolor[rgb]{ .518,  .592,  .69}G                &               \cellcolor[rgb]{ .788,  .788,  .788}D                & \cellcolor[rgb]{ .788,  .788,  .788}D &                  \cellcolor[rgb]{ 1,  .851,  .4}C                  &               \cellcolor[rgb]{ .663,  .816,  .557}A                \\
	    	                       38                         & \cellcolor[rgb]{ .788,  .788,  .788}D &                \cellcolor[rgb]{ .518,  .592,  .69}G                &               \cellcolor[rgb]{ .557,  .663,  .859}F                & \cellcolor[rgb]{ .788,  .788,  .788}D &               \cellcolor[rgb]{ .788,  .788,  .788}D                &                \cellcolor[rgb]{ .957,  .69,  .518}E                \\
	    	                       39                         & \cellcolor[rgb]{ .663,  .816,  .557}A &                \cellcolor[rgb]{ .957,  .69,  .518}E                &               \cellcolor[rgb]{ .663,  .816,  .557}A                & \cellcolor[rgb]{ .663,  .816,  .557}A &               \cellcolor[rgb]{ .608,  .761,  .902}B                &               \cellcolor[rgb]{ .788,  .788,  .788}D                \\
	    	                       40                         & \cellcolor[rgb]{ .663,  .816,  .557}A &                \cellcolor[rgb]{ .957,  .69,  .518}E                &               \cellcolor[rgb]{ .608,  .761,  .902}B                & \cellcolor[rgb]{ .608,  .761,  .902}B & \cellcolor[rgb]{ .251,  .251,  .251}\textcolor[rgb]{ 1,  1,  1}{I} & \cellcolor[rgb]{ .251,  .251,  .251}\textcolor[rgb]{ 1,  1,  1}{I} \\
	    	                       41                         & \cellcolor[rgb]{ .608,  .761,  .902}B &                \cellcolor[rgb]{ .518,  .592,  .69}G                &                \cellcolor[rgb]{ .518,  .592,  .69}G                & \cellcolor[rgb]{ .608,  .761,  .902}B & \cellcolor[rgb]{ .251,  .251,  .251}\textcolor[rgb]{ 1,  1,  1}{I} &                \cellcolor[rgb]{ .518,  .592,  .69}G                \\
	    	                       42                         & \cellcolor[rgb]{ .608,  .761,  .902}B &                \cellcolor[rgb]{ .518,  .592,  .69}G                &                \cellcolor[rgb]{ .957,  .69,  .518}E                & \cellcolor[rgb]{ .788,  .788,  .788}D &               \cellcolor[rgb]{ .557,  .663,  .859}F                &               \cellcolor[rgb]{ .557,  .663,  .859}F                \\
	    	                       43                         & \cellcolor[rgb]{ .788,  .788,  .788}D &                  \cellcolor[rgb]{ 1,  .851,  .4}C                  &               \cellcolor[rgb]{ .788,  .788,  .788}D                &   \cellcolor[rgb]{ 1,  .851,  .4}C    &               \cellcolor[rgb]{ .557,  .663,  .859}F                &                  \cellcolor[rgb]{ 1,  .851,  .4}C                  \\
	    	                       44                         & \cellcolor[rgb]{ .788,  .788,  .788}D &                  \cellcolor[rgb]{ 1,  .851,  .4}C                  & \cellcolor[rgb]{ .251,  .251,  .251}\textcolor[rgb]{ 1,  1,  1}{I} & \cellcolor[rgb]{ .788,  .788,  .788}D &                \cellcolor[rgb]{ .957,  .69,  .518}E                & \cellcolor[rgb]{ .251,  .251,  .251}\textcolor[rgb]{ 1,  1,  1}{I} \\
	    	                       45                         & \cellcolor[rgb]{ .663,  .816,  .557}A &               \cellcolor[rgb]{ .557,  .663,  .859}F                &               \cellcolor[rgb]{ .557,  .663,  .859}F                & \cellcolor[rgb]{ .788,  .788,  .788}D &                \cellcolor[rgb]{ .957,  .69,  .518}E                &                \cellcolor[rgb]{ .957,  .69,  .518}E                \\
	    	                       46                         & \cellcolor[rgb]{ .663,  .816,  .557}A &               \cellcolor[rgb]{ .557,  .663,  .859}F                &               \cellcolor[rgb]{ .663,  .816,  .557}A                & \cellcolor[rgb]{ .663,  .816,  .557}A &               \cellcolor[rgb]{ .608,  .761,  .902}B                &               \cellcolor[rgb]{ .663,  .816,  .557}A                \\
	    	                       47                         &   \cellcolor[rgb]{ 1,  .851,  .4}C    &               \cellcolor[rgb]{ .608,  .761,  .902}B                &                  \cellcolor[rgb]{ 1,  .851,  .4}C                  & \cellcolor[rgb]{ .663,  .816,  .557}A &               \cellcolor[rgb]{ .608,  .761,  .902}B                &               \cellcolor[rgb]{ .788,  .788,  .788}D                \\
	    	                       48                         &   \cellcolor[rgb]{ 1,  .851,  .4}C    &               \cellcolor[rgb]{ .608,  .761,  .902}B                & \cellcolor[rgb]{ .251,  .251,  .251}\textcolor[rgb]{ 1,  1,  1}{I} & \cellcolor[rgb]{ .663,  .816,  .557}A &               \cellcolor[rgb]{ .788,  .788,  .788}D                &                \cellcolor[rgb]{ .518,  .592,  .69}G                \\
	    	                       49                         & \cellcolor[rgb]{ .663,  .816,  .557}A &               \cellcolor[rgb]{ .788,  .788,  .788}D                &               \cellcolor[rgb]{ .788,  .788,  .788}D                & \cellcolor[rgb]{ .663,  .816,  .557}A &               \cellcolor[rgb]{ .788,  .788,  .788}D                &               \cellcolor[rgb]{ .557,  .663,  .859}F                \\
	    	                       50                         & \cellcolor[rgb]{ .663,  .816,  .557}A &               \cellcolor[rgb]{ .788,  .788,  .788}D                &               \cellcolor[rgb]{ .608,  .761,  .902}B                & \cellcolor[rgb]{ .608,  .761,  .902}B &                \cellcolor[rgb]{ .518,  .592,  .69}G                &                \cellcolor[rgb]{ .518,  .592,  .69}G                \\
	    	                       51                         & \cellcolor[rgb]{ .608,  .761,  .902}B &                \cellcolor[rgb]{ .518,  .592,  .69}G                &                \cellcolor[rgb]{ .518,  .592,  .69}G                & \cellcolor[rgb]{ .608,  .761,  .902}B &                \cellcolor[rgb]{ .518,  .592,  .69}G                &                \cellcolor[rgb]{ .957,  .69,  .518}E
	    \end{tabular}%
	\end{adjustbox}
	\caption{Comparison of schedules for 2018}
	\label{tbl:2018-schedule-comparison}%
\end{table}%


In order to evaluate the generated schedule and compare it with the manually created schedule, we outline the adherence of each schedule to the constraints presented in section \ref{???}. As we can see from table \ref{tbl:constraints-comparison}, the generated schedule was able to satisfy all mandatory constraints, however for the years of [???] it could not find an optimal schedule while ensuring that the block assignments are spread out ([???] constraint). On the other hand, we can see that the manually created schedule was not able to satisfy all mandatory constraints. In particular, we see that it contains consecutive block assignments for the years of [???], [...]. Evaluating the objectives, we see that [...\textit{request conflicts}]. Moreover, the manual schedule does not attempt to align weekend assignments with block assignments, unlike the optimal schedule found by the software.

% Table generated by Excel2LaTeX from sheet 'constraints'
\begin{table}[h]
	\centering
    \begin{tabular}{l|cc|cc|cc|cc}
    	\hline
    	\multicolumn{1}{c|}{\multirow{2}[1]{*}{\textbf{Constraint}}} & \multicolumn{2}{c|}{\textbf{2015}} & \multicolumn{2}{c|}{\textbf{2016}} & \multicolumn{2}{c|}{\textbf{2017}} & \multicolumn{2}{c}{\textbf{2018}} \\
    	                                                    & LP &      Historical      & LP &      Historical      & LP &      Historical      & LP &     Historical      \\ \midrule
    	Block Coverage                                      & \checkmark  &                      & \checkmark  &                      & \checkmark  &          \checkmark           & \checkmark  &          \checkmark          \\
    	Weekend Coverage                                    & \checkmark  &                      & \checkmark  &                      & \checkmark  &          \checkmark           & \checkmark  &          \checkmark          \\
    	Min/Max                                             & \checkmark  &                      & \checkmark  &                      & \checkmark  &          \checkmark           & \checkmark  &          \checkmark          \\
    	No Consecutive Blocks                               & \checkmark  &                      & \checkmark  &                      & \checkmark  &                      & \checkmark  &                     \\
    	No Consecutive Weekends                             & \checkmark  &                      & \checkmark  &                      & \checkmark  &          \checkmark           & \checkmark  &          \checkmark          \\
    	Equal Weekends                                      & \checkmark  &                      & \checkmark  &                      & \checkmark  &                      & \checkmark  &                     \\
    	Equal Holidays                                      & \checkmark  &                      & \checkmark  &                      & \checkmark  &                      & \checkmark  &                     \\
    	\hline
    	Block Requests                                      &    &                      &    &                      &    &                      &    &                     \\
    	Weekend Requests                                    &    &                      &    &                      &    &                      &    &                     \\
    	Block-Weekend Adjacency                             &    &                      &    &                      &    &                      &    &
    \end{tabular}%
	\caption{Comparison of constraint satisfaction in LP and Historical schedules}
	\label{tbl:constraints-comparison}%
\end{table}%


\subsection{Simulations}
In this section, we want to analyze properties of the LP problem on simulated data. Our goal is to see whether the aforementioned formulation of the problem is going to suffer from run-time issues, when presented with a large set of constraints arising from a hypothetical department with many clinicians, or clinicians with many time-off requests throughout the year. We also want to analyze the behaviour of the LP problem when applied to a longer time horizon than just a single year.