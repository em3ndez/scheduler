In this section, we present an application of integer linear programming to solve the clinician
% JK: there are a few variations of
%     "nurse scheduling problem", "scheduling problem" "clinician scheduling problem"
%     floating around. Can you do a consistency check throughout?
scheduling problem presented in Section~\ref{sec:problem}. 
An integer linear program (ILP) consists of a linear objective function and linear constraints,
with integer-valued variables. The optimal solution to the ILP must lie within the space
defined by the constraints while also maximizing the objective value.
First, we describe
% JK: Also, while the intro gives a reference for ILP,
%     I think you should define it here. e.g. It is not clear to the naive reader (me!)
%     that ILP needs an objective function
%     or (most importantly) how ILP works to satisfy multiple constraints.
the sets, indices and variables necessary for the formulation of the problem. We
then write the constraints given in Table~\ref{tbl:constraint-summary} as
linear functions of the variables, and define the objective function of the ILP.

\subsection{Sets and Indices}\label{sec:meth-sets-indices}
We denote the set of all services %or I think services is good (vs.
%services/divisions) as you define two types of services (ID vs. HIV) in the
%problem statement
% JK: same thing, consistency check with division / service
%     I think maybe you already did this...
that clinicians in a single department can provide as $\mathcal{S}$. 
%The following formulation assumes that all clinicians are able to provide all services. 
% JK: but could you not also just set some clinician-service max-mins zero
%     to deal with if some clinicians are not able to provide some services?
The set of all clinicians in the department is denoted as
$\mathcal{C}$. The sets of blocks and weekends
% JK: removed "will be assigned to" since on first read I confused it with
%     the binary assignment variable X below.
are denoted as $\mathcal{B}$ and $\mathcal{W}$ respectively. The block size
used in our experiments is 2 weeks, but the following LP formulation does not
require a particular size for blocks, and so it can be adapted for other cases.
%The size of a block is not constrained and can be adapted %how?
%to the needs of the given department. 
A subset of weekends are denoted $\mathcal{L} \subset \mathcal{W}$, corresponding to established
long/holiday weekends such as the Thanksgiving weekend. Lastly,
% JK: I'm being picky, but maybe a non-Canadian holiday like Thanksgiving as the e.g.
the block and weekend time-off requests of clinicians are denoted as the following
subsets of all blocks and weekends: $\mathcal{U}_c \subset \mathcal{B}$ and $\mathcal{V}_c \subset \mathcal{W}$, respectively.
% JK: keeping "x is denoted as x" sentence structure
% JK: Also, I'm generally not a fan of multiple-letter variables since they can be 
%     easily confused for multiplication. Can you find another notation?
For instance, if clinician $c$'s time-off requests intersect with blocks 1 and 2, and weekend
1, then $\mathcal{U}_c = \{1, 2\}$ and $\mathcal{V}_c = \{1\}$. Table~\ref{tbl:sets-indices} presents a summary of the sets and indices described. 

\begin{table}[h]
	\centering
  \caption{Description of sets and indices in the problem}%
  \label{tbl:sets-indices}
	\begin{tabular}{ c c l }
		\toprule
		\textbf{Set}                         & \textbf{Index} & \textbf{Description}  
		\\ \midrule
		$\mathcal{S} = \{1, \ldots, S \}$    & $s$            & services              
		\\
		$\mathcal{C} = \{1, \ldots, C \}$    & $c$            & clinicians            
		\\
		$\mathcal{B} = \{1, \ldots, B \}$    & $b$            & blocks                
		\\
		$\mathcal{W} = \{1, \ldots, W \}$    & $w$            & weekends              
		\\
		$\mathcal{L} \subset \mathcal{W}$    &                & long weekends         
		\\
		$\mathcal{U}_c \subset \mathcal{B}$ &                & block requests of
		clinician $c$   \\
		$\mathcal{V}_c \subset \mathcal{W}$ &                & weekend requests of
		clinician $c$ \\
    \bottomrule
	\end{tabular}
	
\end{table}

\subsection{Variables}\label{sec:meth-variables}
Since each clinician may be assigned to work on any service, during any block of
the year, we denote such an assignment as a binary variable $X_{c, b, s} \in \{0,1\}$.
% JK: I know its implied but if its easy to spell it out might as well.
A value of 1 indicates that clinician $c$ is assigned to service $s$ during block $b$,
while a value of 0 indicates they are not assigned.
Weekend assignments are similarly defined using a binary
variable $Y_{c, w} \in \{0,1\}$, but without a service index, as clinicians are expected to
provide all services during the weekends. We then define
$m_{c, s}$ and $M_{c, s}$ to represent the minimal and maximal number of blocks of service $s$
that clinician $c$ is required to work during the year.
Table~\ref{tbl:variables-constants} presents a summary of the constants and variables
in the problem.
%To optimize the soft constraint Block-Weekend Adjacency, we maximize the
%product $X_{c, b, s} \cdot Y_{c, w}$ for adjacent blocks and weekends. To
%formulate such an objective as a linear function of variables, we introduce
%another set of variables, denoted by $Z_{c, b, s}$, with additional constraints
%on its range. Further details regarding this variable are described in Section
%\ref{sec:meth-objectives}. 

\begin{table}[h]
	\centering
  \caption{Description of variables and constants in the problem}%
  \label{tbl:variables-constants}
	\begin{tabular}{ c l }
		\toprule
		\textbf{Name}              & \textbf{Description}                             
		\\ \midrule
		$X_{c, b, s} \in \{0, 1\}$ & assignment of clinician $c$ to service $s$ on
		block $b$            \\
		$Y_{c, w} \in \{0, 1\}$    & assignment of clinician $c$ on weekend $w$       
		\\
		%		$Z_{c, b, s} \in \{0, 1\}$ & helper variable for optimizing Block-Weekend
		%adjacency             \\
		$m_{c, s}$                 & minimum number of blocks clinician $c$ should
		cover on service $s$ \\
		$M_{c, s}$                 & maximum number of blocks clinician $c$ should
		cover on service $s$ \\
    \bottomrule
	\end{tabular}
\end{table}

\subsection{Constraints}\label{sec:meth-constraints}
We now formalize the hard constraints in Table~\ref{tbl:constraint-summary}
using the variables defined above.

% JK: not sure how I feel about these letters as equation refs
\begin{align}
&\sum_{c=1}^{C} X_{c, b, s} = 1 &&\forall b\in \mathcal{B}, s \in \mathcal{S}
\tag{BC} \label{eqn:constr-block-cov} \\
&\sum_{c=1}^{C} Y_{c, w} = 1 &&\forall w\in \mathcal{W} \tag{WC}
\label{eqn:constr-weekend-cov} \\
&m_{c, s} \leq \sum_{b=1}^{B} X_{c, b, s} \leq M_{c, s} &&\forall\
c\in\mathcal{C}, s\in\mathcal{S} \tag{MM} \label{eqn:constr-min-max} \\
&X_{c, b, s} + X_{c, b + 1, s} \leq 1 &&\forall c\in\mathcal{C}, b \leq B - 1,
s\in\mathcal{S} \tag{NCB} \label{eqn:constr-no-consec-blocks} \\
&Y_{c, w} + Y_{c, w + 1} \leq 1 &&\forall c\in\mathcal{C}, w \leq W - 1
\tag{NCW} \label{eqn:constr-no-consec-weekends} \\
&\floor*{\frac{W}{C}} \leq \sum_{w=1}^W Y_{c, w} \leq \ceil*{\frac{W}{C}}
&&\forall c\in\mathcal{C} \tag{EW} \label{eqn:constr-equal-weekends} \\
&\floor*{\frac{\abs{\mathcal{L}}}{C}} \leq \sum_{w\in\mathcal{L}} Y_{c, w} \leq
\ceil*{\frac{\abs{\mathcal{L}}}{C}} &&\forall c\in\mathcal{C} \tag{EH}
\label{eqn:constr-equal-holidays}
\end{align}

\subsection{Objectives}\label{sec:meth-objectives}
As described in Section~\ref{sec:problem}, the soft constraints of the clinician
scheduling problem include: satisfying clinician block off requests (BR),
satisfying clinician weekend off requests (WR), and assigning weekends closer to
blocks (BWA). We convert these soft constraints into linear objective functions
of the binary variables defined in Section~\ref{sec:meth-variables}. Objectives
(\ref{eqn:obj-block-requests}) and (\ref{eqn:obj-weekend-requests}) can be
written as the following linear functions of $X$ and $Y$:  %minor suggestion - consider
%numbering the objectives for ease later when saying things like "these two
%objectives"...or writing the full name of the objectives instead of 'this' and
%'these' as sometimes can get confusing what the 'this' and the 'these' refer
%to. 
\begin{align}
&Q_1(X) = \sum_{c=1}^{C} \sum_{b=1}^{B} \sum_{s=1}^{S}
{(-1)}^{\ind(b\,\in\,\mathcal{U}_c)}\cdot X_{c, b, s} \tag{BR}
\label{eqn:obj-block-requests}\\
&Q_2(Y) = \sum_{c=1}^{C} \sum_{w=1}^{W}
{(-1)}^{\ind(w\,\in\,\mathcal{V}_c)}\cdot Y_{c, w} \tag{WR}
\label{eqn:obj-weekend-requests}
\end{align}
where $\ind(P)$ is the indicator function that has value 1 when predicate $P$
holds and 0 otherwise. In the above two objectives, we penalize any assignments
that conflict with a block or weekend request, and aim to maximize the
non-conflicting assignments.

The Block-Weekend Adjacency is optimized by considering the product $X_{c, b,
	s}\cdot Y_{c, w}$ for values of $w$ ``adjacent'' to the value of $b$. This
leads
to the maximization objective %avoid editorial adjectives
\begin{align}
&Q_3(X, Y) = \sum_{c=1}^{C} \sum_{b=1}^{B} \sum_{s=1}^{S} X_{c, b, s}\cdot Y_{c,
	w=\varphi(b)} \tag{BWA} \label{eqn:obj-block-weekend-adj}
\end{align}
where $\varphi(b)$ is a one-to-one mapping of a block to an adjacent weekend, by
some appropriate definition of adjacency. For instance, clinicians might want to
be assigned during a weekend that falls within an assigned block. In this case,
we will have $\varphi(b) = 2b-1$.

%So if clinician $c$ is assigned to work during block 3, corresponding to weeks
%5 and 6 assuming 2-week blocks, they might also want to be assigned to work
%during weekend 5. In that case, we would like $X_{c, b=3, s} \cdot Y_{c, w=5}$
%to be 1, since that indicates both variables are assigned. If at least one of
%the two variables is not assigned, the product will be 0. 
% I really liked the example explanation!
%For instance, in the above example we will have $\varphi(b) = 2b - 1$. \\

However, as it is, $Q_3$ is not a linear function of the assignment variables
$X$ and $Y$, and cannot be optimized in a linear programming framework. 
An approach used to convert such
functions into linear objectives involves introducing a helper variable and
additional constraints~\cite{hammer_boolean_1968}. 
In our case, we introduce a variable $Z_{c, b, s}$ 
for every product $X_{c, b, s} \cdot Y_{c, w}$ with $w = \varphi(b)$, and
constraining $Z$ such that 
\begin{align}
&Z_{c, b, s} \leq X_{c, b, s} \label{eqn:helper-x-constraint}\\
&Z_{c, b, s} \leq Y_{c, w=\varphi(b)} &&\forall s\in\mathcal{S}
\label{eqn:helper-y-constraint}
\end{align}
Since $X_{c, b, s}$ and $Y_{c, w}$ are binary variables, $Z_{c, b, s}$ will be constrained
to 0, unless both $X_{c, b, s}$ and $Y_{c, w}$ are 1. Therefore, it suffices to maximize
the following linear function of $Z$,
\begin{align}
&Q_3(Z) = \sum_{c=1}^{C} \sum_{b=1}^{B} \sum_{s=1}^{S} Z_{c, b, s}
\end{align}
to get the correct adjacency maximization objective.
%In our case, introducing a
%variable $Z_{c, b, s}$ for every product $X_{c, b, s} \cdot Y_{c, w}$ with $w =
%\varphi(b)$, and constraining $Z$ such that 
%\begin{align}
%&Z_{c, b, s} \leq X_{c, b, s} \label{eqn:helper-x-constraint}\\
%&Z_{c, b, s} \leq Y_{c, w=\varphi(b)} &&\forall s\in\mathcal{S}
%\label{eqn:helper-y-constraint}
%\end{align}
%allows us to rewrite $Q_3$ as a linear function of $Z$,
%\begin{align}
%&Q_3(Z) = \sum_{c=1}^{C} \sum_{b=1}^{B} \sum_{s=1}^{S} Z_{c, b, s}
%\end{align}
%Indeed from Eqns. (\ref{eqn:helper-x-constraint}) and
%(\ref{eqn:helper-y-constraint}),  whenever $X_{c, b, s} \cdot Y_{c, w} = 1$
%(respectively, 0), $Z_{c, b, s}$ can attain a maximum value of 1 (respectively,
%0), giving us the correct adjacency maximization objective.
% JK: not sure I follow these (respectively, 0).
%Indeed, whenever $X_{c, b, s} \cdot Y_{c, w} = 1$, $Z_{c, b, s}$ can attain a
%maximum value of 1, and whenever $X_{c, b, s} \cdot Y_{c, w} = 0$, at least one
%of $X_{c, b, s}$ or $Y_{c, w}$ must be 0, so $Z_{c, b, s}$ will be constrained
%to attain a maximum value of 0, giving us the correct adjacency maximization
%objective. \\

In order to optimize all objectives simultaneously, we optimize a weighted sum
of the normalized objective functions,
% JK: This only considers the soft constraints.
%     Should this maximization not also be subject to the Constraints in 3.3?
\begin{equation}
\max_{X, Y, Z} \alpha_1 \bar{Q}_1(X) + \alpha_2 \bar{Q}_2(Y) + \alpha_3
\bar{Q}_3(Z)
\end{equation}
% JK: I would use $\alpha_1, \alpha_2, \alpha_3$ here which could take any value,
%     and then not necessary to have any bounds like \sum_i^3 \alpha_i = 1,
%     though you could enforce it to compare across permutations of \alpha
subject to the constraints defined in Section \ref{sec:meth-constraints},
where $\bar{Q}_i$ is the normalization of objective $Q_i$, and $0 \leq \alpha_i \leq 1$. 
This method guarantees an optimal solution to be Pareto optimal~\cite{stanimirovic_linear_2011}. 

Currently, the most efficient approach to finding an exact solution for 
an ILP is called Branch-and-Cut~\cite{mitchell_branch-and-cut_2002}. 
This method involves iteratively solving 
LP relaxations versions of the ILP, then constraining the relaxed problems and
considering various sub-problems until it finds integral solutions to the 
original ILP.
The intermediate relaxations allow the variables to take on real values, 
and can be solved efficiently using the Simplex method~\cite{shamir_efficiency_1987}. The 
complexity of finding an optimal integral solution thus lies in the 
branching search structure of the approach.

% JK: How can ILP be used solve such problems? Can you briefly describe?

%Thus, our clinician scheduling problem is a multiple objective optimization
%problem. The most common approach to solving multiple objective optimization
%problems is by optimizing a weighted sum of the normalized objective functions,
%as this guarantees the optimal solution to be Pareto optimal [ref \ref{???}].
%This is the approach we decided to use in our problem, to ensure all three
%objectives are considered when finding a solution. Under the assumption that
%each clinician in $\mathcal{C}$ provides all types of services in
%$\mathcal{S}$, the normalized objectives can be written as follows,
%\begin{align}
%	&\bar{Q}_1(X) = \frac{Q_1(X)}{C \cdot B \cdot S} \tag{Block Requests}
%\label{eqn:norm-obj-block-requests}\\
%	&\bar{Q}_2(Y) = \frac{Q_2(Y)}{C \cdot W} \tag{Weekend Requests}
%\label{eqn:norm-obj-weekend-requests} \\
%	&\bar{Q}_3(Z) = \frac{Q_3(Z)}{C \cdot B \cdot S} \tag{Block-Weekend Adjacency}
%\label{eqn:norm-obj-block-weekend-adj}
%\end{align}
%where we divide each of the original objective functions by the sum of the
%absolute values of its coefficients [ref \ref{???}]. The final weighted
%objective is given by
%\begin{equation}
%	\alpha \bar{Q}_1(X) + \beta \bar{Q}_2(Y) + (1 - \alpha - \beta) \bar{Q}_3(Z)
%\end{equation}
%with $0 \leq \alpha, \beta \leq 1$. \\

