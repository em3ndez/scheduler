Scheduling of personnel in a hospital environment is vital to improving the service provided to patients and balancing the workload assigned to clinicians. Many approaches have been tried and successfully applied to generate efficient schedules in such settings. However, due to the computational complexity of the scheduling problem in general, most approaches resort to heuristics to find a non-optimal solution in a reasonable amount of time. Our paper describes an integer linear programming formulation to finding an optimal schedule in a clinical division of a hospital. It mitigates computational complexity issues by maintaining a minimal set of constraints, yet still provides the flexibility necessary to adapt the formulation to a variety of clinical divisions. \\

A case study for our approach is presented, using data from the Infectious Diseases division at St. Michael's Hospital in Toronto, Canada. We analyze and compare the results of our approach to manually-created schedules at the hospital, and note much improved adherence to departmental constraints and clinician preferences. We also perform runtime analysis of our linear program for various parameters and observe reassuring results, signifying the practicality of our approach in different real-world scenarios.