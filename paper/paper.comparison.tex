In recent years, there has been a substantial amount of literature published in
the nurse rostering area aiming to organize the types of problems and approaches
commonly
addressed in the real-world. This research serves to highlight the similarities
and differences
among computationally easy and hard instances of nurse rostering, evaluating
which algorithms
are best geared towards which types of problems, and [...]

De Causmaecker and Berghe [ref] outlined a categorization
of nurse rostering problems using a similar approach used in personnel
scheduling,
based around the personnel environment, work characteristics and optimization
objective. [...]

[a few more papers on categorization, brief paragraph about each...]

We follow the work in [ref-De Causmaecker and Berghe] to place our clinician
scheduling problem
in the realm of work for nurse rostering. According to their specification, our
problem includes 
personnel environment constraints targeting the availability of clinicians
([list which constraints...]),
sequences of assignments ([...]), and balance of rosters ([...]) 
which fall under the categories $\alpha : A$, $\alpha : S$ and $\alpha : B$,
respectively. 
For work characteristics, the relevant category for our problem are the range
constraints, $\alpha : R$, 
as we restrict the workload of each clinician between a minimum and maximum
allowed blocks ([constraint...]).
Lastly, for the optimization objective our problem corresponds to the category
$\gamma : P$, as our optimization objective
aim to adhere to clinician preferences. Therefore the appropriate categorization
for our problem is $ASB/R/P$.

[outline and compare to papers with a similar categorization...]
