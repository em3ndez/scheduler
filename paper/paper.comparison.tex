%Prior work by [authors] have   %avoid editorial type statements without citations or numbers, best to state as objective statements
%provided categories to organize the types of problems and approaches
%commonly used for real-world nurse rostering.
%These categories highlight the similarities
%and differences
%among computationally easy and hard instances of nurse rostering and 
%help end-users determine
%which algorithms
%are best suited to each type of problem.

In 2010, De Causmaecker and Berghe~\cite{de_causmaecker_categorisation_2011} addressed a gap in the literature 
surrounding a central categorization system for the nurse rostering problems identified in the literature.
The authors created a categorization framework that takes into account
hard and soft constraints related to nurse preferences, consecutive
assignments, coverage and various flexibilities in terms of shift and 
skill types. The described framework is inspired by the categorization used
in theoretic scheduling problems which focuses on three main aspects of scheduling:
(1) the machine environment;
(2) the job characteristics;
and (3) the objective function.
De Causmaecker and Berghe use these three aspects as a basis for their categorization,
and adjust them to fit the realm of nurse rostering. A description of the categories defined
by the authors is given in Table \ref{tbl:nurse-rostering-categorization}.

\begin{table}[h]
	\centering
	\caption{Description of De Causmaecker and Berghe's categorization system for nurse rostering problems.}%
	\label{tbl:nurse-rostering-categorization}
	\begin{tabular}{l|l|l}
		Category                                          & \multicolumn{2}{l}{Sub-categories}                                                                                                                                                         \\ \hline
		\multirow{2}{*}{$\alpha$: Personnel environment}  & Personnel Constraints                                                                                      & Skill Interactions                                                            \\ \cline{2-3}
		                                                  & \makecell[l]{(A) Availability \\ (S) Sequences \\ (B) Balance \\ (C) Chaperoning}                                          & \makecell[l]{Fixed number \\ Variable number \\ Individual skill definitions} \\ \hline
		\multirow{2}{*}{$\beta$: Work characteristics}    & Coverage Constraints                                                                                       & Shift Type                                                                    \\ \cline{2-3}
		                                                  & \makecell[l]{(R) Range \\ (T) Time intervals \\ (V) Fluctuating}                                                       & \makecell[l]{Fixed number \\ Variable number \\ Overlapping}                  \\ \hline
		\multirow{2}{*}{$\gamma$: Optimization objective} & Objective                                                                                                  & Mode                                                                          \\ \cline{2-3}
		                                                  & \makecell[l]{(P) Personnel constraints \\ (L) Coverage constraints\\ (X) Number of personnel \\ (R) Robustness \\ (G) General} & \makecell[l]{Multi-objective}                                                 \\ \hline
	\end{tabular}
\end{table}

% DL - maybe this should be in complexity section? %SM: agree, move to complexity section.
%Vanhoucke and Maenhout [ref] analyze and compare nurse rostering instances
%based on complexity indicators arising from problem constraints. They consider
%problem size (in terms of number of nurses, shifts and days), preference distribution
%of the nurses, coverage distribution and time-related constraints (such as weekend
%and consecutive constraints) as the four indicators for problem complexity.
%Then, they generate sample problems corresponding to all four complexity indicators
%and measure the effect of increasing complexity on CPU time required for solving each
%problem. 

We follow the categorization in Table \ref{tbl:nurse-rostering-categorization} to situate our clinician
scheduling problem. Our problem includes 
personnel environment constraints targeting the availability and preferences of clinicians
(constraints BR and WR in Table \ref{tbl:constraint-summary}),
sequences of assignments (constraints NCB and NCW), and balance of rosters (constraints EW and EH) 
which fall under the categories $\alpha : A$, $\alpha : S$ and $\alpha : B$,  %are these the categories? think other way around -- these fall into the named categories from De Causeameker, etc.?
respectively. 
For work characteristics, the relevant category for our problem are the range
constraints, $\beta : R$, 
as we restrict the workload of each clinician between a minimum and maximum
allowed blocks (constraint MM).
Lastly, for the optimization objective our problem corresponds to the category
$\gamma : P$, as our optimization objective
aims to adhere to clinician preferences. Therefore the appropriate categorization
for our problem is $ASB/R/P$.

De Causmaecker and Berghe also present examples from the nurse rostering literature
and their appropriate categorization in their framework. We use some of these examples
to compare our clinician
scheduling problem and solution approach with existing literature. Azaiez and Al Sharif~\cite{azaiez_0-1_2005}
describe a nurse rostering problem to schedule nurses on a per-day basis, for a total period
of 28 days. Azaiez and Al Sharif's problem is categorized as $ASBN/V2/PL$.
Similar to our clinician scheduling problem, the authors include hard constraints
to prevent consecutive assignments, and enforce a minimum and maximum number of shifts for each nurse.
Unlike our problem, Azaiez and Al Sharif choose an optimization goal that focuses on
balancing the workload among nurses, preventing day-night consecutive shifts and 
avoiding off-on-off assignments. The authors do not attempt to incorporate nurse preferences
for time off into the model. 

In~\cite{bilgin_local_2012}, Bilgin et al address a nurse rostering problem using local search.
De Causmaecker and Berghe categorize the problem in~\cite{bilgin_local_2012} as $ASBI/RVNO/PLR$.
Bilgin et al incorporate coverage constraints, overlapping assignment constraints
and horizontal constraints, which include constraints arising from employment contracts
(such as minimum and maximum thresholds).
Unlike our approach for these constraints, the authors consider coverage constraints to be soft
constraints, in order to prevent over-constraining of the problem.
Bilgin et al also incorporate nurse requests into the model formulation by allowing
nurses to request days off and days on.

Burke et al~\cite{burke_metaheuristics_2006} present a flexible framework for nurse rostering based on various approaches
observed from Belgin hospitals. 
In~\cite{de_causmaecker_categorisation_2011}, Burke et al's work is categorized as $ASBCI/RVNTO/PR$.
The authors use a two-step algorithm comprised of an initialization procedure
which provides a solution satisfying only the hard constraints of their problem,
and a metaheuristic procedure, a hybrid tabu search, which aim to satisfy as many soft constraints as possible.
Unlike our clinician scheduling problem, 
Burke et al's hard constraints address only coverage requirements in an effort to make the first
step of their algorithm efficient.
In their metaheurstic procedure, the authors optimize soft constraints including assignment of free weekends
and minimum number of assignments per staff member.
Notably, Burke et al's approach make use of ``time interval personnel requirements'', which describe
the amount of required staff per skill category as a varying number throughout the work day.