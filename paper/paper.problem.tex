Clinicians at the Infectious Diseases (ID) and HIV departments of St. Michael's Hospital are scheduled throughout the year to receive patients during on-call hours. Each clinician is typically scheduled for a full week or a full weekend of on-call service. To prevent under- and over-working of clinicians, they each have a minimum and maximum number of weeks that they are required to work. Moreover, during holidays weekends, the work-load for on-call service increases drastically, and it is important to distribute these weekends equally. At the same time, clinicians often request to not be put on service during certain days or weeks, and the generated schedule should attempt to respect those requests as best as possible. Other considerations also need to be taken into account, such as making sure someone is available for on-call service at any time of the year, and preventing multiple back-to-back assignments for a single clinician. It is clear that ensuring all of these conditions are met manually, especially with an increasing number of clinicians, is a difficult task and can lead to mismanagement of the schedule. Therefore, in this paper we attempt to develop an efficient algorithm that can generate a satisfying schedule, while optimizing clinician and patient contentment. \\

In order to develop an algorithm, we need to formalize the variables and constraints of the problem mathematically. Table [??] presents the sets and indices that are used in the definition of the constraints. [...] \\

[...]