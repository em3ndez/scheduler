In this paper our goal was to develop a rather simple, yet flexible, integer linear programming formulation to generate schedules for clinical consultation (?) departments at hospitals. The difficulty in applying a linear programming approach to this task lies in the fact that ILP is an NP-hard problem, and as such the time to find an optimal solution grows exponentially as we increase the size of the problem. As a result, the vast majority of approaches taken to create schedules for similar scenarios tend to use heuristics in order to find an approximately optimal solution in a shorter time. \\

Our formulation includes hard constraints to ensure the schedule satisfies hospital and logistics requirements. It also aims to satisfy the work preferences of clinicians in a clinical department by optimizing a multi-goal objective function. We are able to use the tool to generate schedules for a wide variety of simulated departments, supporting upwards of 50 total clinicians. Moreover, our formulation is able to accommodate the requests of clinicians without negatively affecting the runtime. The scenarios when the ILP formulation starts to have trouble are ones with multiple services offered within a single clinical department. Once there are more than 10 clinicians in total, providing services in 2 or more divisions, the constraint space becomes very complex, making it difficult for the tool to find an optimal solution in a reasonable amount of time. For these cases, we recommend allowing consecutive blocks, to simplify the problem and find a solution much faster. \\

Our tool is also employed at the Infectious Diseases Department at St. Michael's Hospital. It provides a simple and easy-to-use user interface that can be used by any hospital staff to generate schedules well in advance. We compare the manually created schedule at St. Michael's Hospital to the optimal schedule generated with the help of our tool. We see adherence to all required hard constraints in the generated schedule, while some of them were overlooked in the manual schedule. Moreover, we see a much better fulfillment of clinician requests than in the manual schedule. This reinforces the idea that it is vital to employ automated tools when generating schedules in hospital departments, to reduce human error, balance the work-load of clinicians and improve the service provided to patients.
