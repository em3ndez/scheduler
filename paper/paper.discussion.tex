In this paper our goal was to develop a rather simple, yet flexible, integer linear programming formulation to generate schedules for clinical departments at hospitals. The difficulty in applying such an approach to this task lies in the fact that ILP is an NP-hard problem, and as such, the time to find an optimal solution grows exponentially as we increase the size of the problem. As a result, the vast majority of approaches taken to create schedules for similar scenarios tend to use heuristics in order to find an approximately optimal solution in a shorter time. \\

Our formulation includes hard constraints to ensure the schedule satisfies hospital and logistics requirements. It also aims to satisfy the work preferences of clinicians in a clinical department by optimizing a multi-goal objective function [...] \\

We developed a tool that implements the ILP formulation and generates an optimal schedule based on provided input data. It provides a simple and easy-to-use user interface that can be used by any hospital staff to generate schedules well in advance. We compared the optimal schedule generated with the help of the tool to the manually-created schedules at St. Michael's Hospital. We found that the ILP formulation was always able to find an optimal schedule satisfying all required hard constraints, unlike the manual schedule, that often did not take all constraints into account. Moreover, due to the multi-goal objective function in the ILP, the tool was able to maximize and fulfill the majority of clinician preferences and requests, more so than the manually-created schedule. These observations reinforce the idea that it is vital to employ automated tools when generating schedules in hospital departments, to reduce human error, balance the work-load of clinicians and improve the service provided to patients. \\

We analyzed the performance of our formulation on simulated data meant to resemble real-life clinician departments at different hospitals. Our results show that increasing the number of requests per clinician does not affect the runtime of the algorithm, highlighting the flexibility of the tool in incorporating clinician preferences. Further, we saw that the algorithm can accommodate increasing the time-horizon up to four years with little impact on runtime, which can be applied to departments that generate schedules far in advance. The most notable impact on the runtime was noticed when trying to increase the number of services offered in a single department. While such cases are unlikely to be encountered in the real-world, as most departments tend to focus on a single service [ref \ref{???}], the runtime issues can be mitigated by relaxing certain constraints. \\

In future work, we can look first and foremost into improving the runtime of the tool by modifying constraints and objectives, so that it can be used in larger departments offering multiple services for their patients. On top of that, we can introduce new objectives that incorporate additional clinician preferences and allow for better distribution of work-load. However, since our ultimate goal is to maintain a ILP formulation, care must be taken not to overly complicate the model and risk running into additional time complexity issues.