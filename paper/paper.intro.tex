On-call schedules for a fixed number of health-care providers are central to the efficient running of hospitals. Hospital departments provide services where patient needs, and thus the system's demands, often exceed the available supply. For example, it is important that a hospital department allocates its resources, such as the availability of a finite number of clinicians, optimally, to ensure the best possible service for its patients. Carefully allocated on-call schedules are meant to simultaneously ensure sufficient resources are provided to patients while not overworking clinicians to prevent costly mistakes [ref]. It is common practice for on-call schedules to be created manually. Yet manually-created schedules are prone to errors and potential for biases [ref]. First, when there is a large number of clinicians in a single department, or the constraints that need to be satisfied by the department are very complex, a manual method may not provide an optimal schedule. Second, such methods are likely to overlook certain constraints that must be maintained to have an operational department, such as XXXX. Third, manual scheduling is often time-consuming for the person developing the schedule. For these reasons, it is important to develop automated methods that can generate optimal schedules that satisfy the given constraints of the hospital department. \\

%When generating these schedules, there are many variables and constraints that need to be taken into account, such as preventing multiple consecutive assignments for a given clinician or ensuring that assignments are spread out evenly throughout a certain period of time. It is very common for departments to opt for a manual method of generating these schedules [??]. Unfortunately, such methods are unreliable and can lead to unfavourable assignments, since a manual method may miss certain constraints or forget to account for a certain variable during the process. \\

Automated methods to optimize schedules have been studied and applied in many industries, including transportation [??], manufacturing [??], [...]. Of special interest to a clinician on-call scheduling problem are the approaches to schedule nurses, who often work in shifts. In the nurse scheduling problem, the goal is to find an optimal assignment of nurses to shifts that satisfies all of the hard constraints, such as hospital regulations, and as many soft constraints as possible, which may include nurse preferences. A wide variety of approaches, including exact and heuristic approaches, have been used to solve the nurse scheduling problem: integer linear programming [??], network flows [??], genetic algorithms [??], simulated annealing [??], and artificial intelligence [??]. 

%An extensive literature review of these and other methods is presented by [??]. We will briefly summarize the main ideas of some of these approaches. \\ SM - don't need to introduce that you will do this for this type of paper I think - would do in a thesis chapter though.

[...] Many of these approaches were designed to satisfy the requirements of a specific hospital department which causes a large number of variables and constraints to be incorporated into the problem formulation. While these department-specific approaches allow end-users to find precise schedules that satisfy the needs of the department and the preferences of the nurses and clinicians in that department, they are difficult to readily adapt to other departments in the same hospital or other hospitals. % explain why hard to adapt?... what makes their generalizablility/adaptability limited?
Moreover, the large number of variables and constraints also leads to computational complexity issues [ref], especially when using exact methods for finding the solution. In this paper, we tackle a version of the nurse scheduling problem arising from a case study of one clinical division, providing two different services simultaneously (general infectious disease consults; and HIV consults service) at St. Michael's Hospital in Toronto, Canada. Our goal is to (1) present a simple formulation for the problem developed and tested at the hospital after switching from a manual approach to scheduling; and (2) analyze the performance of integer linear programming in solving difficult instances of the problem and compare the results with those of the manual approach; and (3) describe the adaptability of the formulation as a basic framework for solving similar problems in other departments. \\ %make #3 an objective
We begin by describing the problem, then...% list the main contents/elements of the paper.

[...]

%Automated methods for scheduling in hospital settings have been studied extensively, most notably in the context of the Nurse Scheduling Problem (NSP) in Operations Research. In an NSP, the goal is to assign nurses to shifts, while satisfying various constraints pertaining to both hospital regulations and nurse preferences. The constraints can be classified as soft or hard constraints, based on the importance of satisfying them. Typically, it is necessary for a solution to satisfy all of the hard constraints, while the soft constraints are considered optional. The NSP, in its general case, is known to be NP-complete, meaning that although it is possible to efficiently check whether a given schedule satisfies all the constraints using an algorithm, it is not yet known whether an efficient algorithm for finding a satisfying schedule exists, and it is in fact equivalent to the hardest computational problems in Computer Science, Operations Research, and \textit{etc...} [??]. In order to tackle NP-complete problems such as NSP in real-life, algorithms either try to use various heuristics to approximate a potential solution rather than finding an exact solution or they focus on relatively small problem sizes, ensuring that the runtime of the algorithm is still practical. \textit{Some examples of heuristic approaches...} . \textit{Some examples of exact approaches (LP, etc.)} . A comprehensive review of various approaches is presented by Burke et al. [??]. \\

%In this paper we present the application of two methods, Network Flows and Integer Linear Programming (ILP), to a variation of the Nurse Scheduling Problem in order to generate a yearly schedule for clinicians working simultaneously at two different divisions of St. Michael's Hospital in Toronto, ON. The paper is organized as follows. Section [??] contains a detailed description of the scenario and a mathematical formalization of the problem. Section [??] describes first the Network Flow approach followed by the ILP approach to tackle the problem. Section [??] presents the results of both approaches when applied to the clinician data from St. Michael's Hospital. Section [??] presents results from simulations with a variety of problem sizes. We conclude the paper with a discussion of the real-world and simulation results in Section [??].