Hospital departments must allocate and use their limited resources efficiently
in order to provide a high quality of care for their patients.
In particular, on-call schedules for
a fixed number of health-care providers are central to the efficient running of
hospitals. Carefully allocated schedules should balance
sufficient staff with workload
to maximize quality of care. It is common for on-call schedules in
hospitals to be created manually. However,
manually-created schedules are subject to three problems.
First, when there are a large number of clinicians,
or the constraints that need to be satisfied
by the schedule are complex, it becomes infeasible
to find a satisfying schedule by hand. 
Second, when creating a schedule manually it is difficult to ensure
that all constraints are met while also trying to satisfy all staff
preferences.
Third, manual scheduling is often time-consuming even for relatively small 
departments, and can take up time and resources that are better used
for improving patient care.
For these reasons, it is important to develop automated methods that
can efficiently generate schedules that satisfy the given constraints.

Automated methods to generate schedules have been studied and applied in many
industries, including
transportation~\cite{aickelin_improved_2006, goel_truck_2012, gunther_combined_2010},
%manufacturing~\cite{al-yakoob_mixed-integer_2007, al-yakoob_column_2008, alfares_simulation_2007},
retail~\cite{chapados_retail_2011, nissen_automatic_2010}, and
military~\cite{horn_scheduling_2007, laguna_modeling_2005}.
Of special interest to a clinician scheduling
problem are the approaches to scheduling nurses, who often work in shifts. In the
nurse scheduling problem, the goal is to find an optimal assignment of nurses to
shifts that satisfies all hard constraints (such as hospital regulations),
and as many soft constraints (such as nurse preferences) as possible.
Hard constraints must be satisfied by any candidate solution to the nurse scheduling problem,
while soft constraints can be used to rank the candidate solutions. 
For instance, a nurse scheduling problem may include a hard constraint to assign
at most a single shift for each nurse per day. It can also incorporate 
nurse preference for shift time (that is, day versus night shifts) as a soft constraint
that is meant to optimize the schedule, but is not guaranteed to be fulfilled.
A wide variety of approaches, including exact and heuristic approaches, have been
used to solve the nurse scheduling problem:
integer linear programming~\cite{azaiez_0-1_2005, trilling_nurse_2006, widyastiti_nurses_2016},
network flows~\cite{el_adoly_new_2018},
genetic algorithms~\cite{aickelin_exploiting_2000, jan_evolutionary_2000, kawanaka_genetic_2001},
simulated annealing~\cite{jaszkiewicz_metaheuristic_1997}, and
artificial intelligence~\cite{abdennadher_nurse_1999, li_hybrid_2003}.
A comprehensive
literature review of these and other methods applied to nurse scheduling is
presented in~\cite{burke_state_2004}.

Many of the approaches to nurse scheduling were designed to satisfy the
requirements of a specific hospital department, which results in a large number of
variables and constraints to be incorporated into the problem formulation. While
these department-specific approaches allow end-users to find precise schedules
that satisfy the needs of that department and the preferences of nurses and
clinicians in that department, they are difficult to adapt to other
departments.
Moreover, the large number of variables and constraints also leads to
computational complexity issues~\cite{goos_complexity_1996}, especially when
trying to find the most optimal solution. In particular, difficult instances of these formulations
become impossible to solve in a reasonable amount of time. 
In this paper, we tackle the clinician 
scheduling problem arising from a case study of one clinical
division, providing two different services (general infectious
disease (ID) consults; and HIV consults) at St.\ Michael's Hospital in
Toronto, Canada. The clinician scheduling problem involves creating
a yearly schedule that assigns clinicians to on-call work on a weekly basis,
while ensuring a fair and balanced workload.
Our goals in this paper are to (1) present an integer linear programming 
(ILP) formulation for our problem, and
describe the flexibility of this formulation for solving similar problems;
(2) compare the performance of our tool for solving the ILP formulation to the
results of a manual approach; 
and (3) analyze the robustness of this approach
in difficult instances of the problem, by exploring the change in runtime with
changes to:
the number of clinicians,
the number of services provided by the department,
the number of requests per clinician, and
the time-horizon of the schedule.

We begin by describing the details of the problem in Section~\ref{sec:problem}, and presenting
our ILP formulation in Section~\ref{sec:methods}. We then categorize our nurse rostering
problem and solution approach in the categorization scheme of~\cite{de_causmaecker_categorisation_2011}
and compare our problem to recent studies in Section~\ref{sec:comparison}.
Next, we analyze the effectiveness of our automated approach when compared
to manually-created schedules, and evaluate the performance
of the algorithm on simulated data in Section~\ref{sec:experiments}. Finally, we
discuss and interpret the results in Section~\ref{sec:discussion}.