Hospital departments provide services where patient needs, and thus the system's
demands, often exceed the available supply. In particular, on-call schedules for
% JK: Not sure I follow this sentence.
%     I think we dont need to say demands > needs, can just say efficieny improves care.
a fixed number of health-care providers are central to the efficient running of
hospitals. Carefully allocated schedules should balance
sufficient staff with clinician workload,
to maximize quality of care. It is common for on-call schedules in
hospitals to be created manually. However,
manually-created schedules are subject to three problems.
First, when there are a large number of clinicians,
or the constraints that need to be satisfied
by the schedule are complex, a manual method may not provide a schedule
that satisfies all constraints. Second, manual methods may overlook
certain constraints that must be maintained to have an operational department,
such as preventing many consecutive work blocks from being assigned or ensuring
clinicians are allocated a specific amount of work blocks throughout the year.
% JK: Can you clarify why / how these may be overlooked?
Third, manual scheduling is often time-consuming.
For these reasons, it is important to develop automated methods that
can efficiently generate schedules that satisfy the given constraints.
% JK: haven't introduced what "optimal" means yet so I think maybe leave it out for now.
%For example, it is important that a hospital department allocates its
%resources, such as the availability of a finite number of clinicians,
%optimally, to ensure the best possible service for its patients.
% JK: instead of using \\ everywhere for paragraph breaks, use:
%     \setlength{\parskip}{1em}
%     in the preamble

Automated methods to generate schedules have been studied and applied in many
industries, including
% JK: I find this is a nice way to organize these lists of topics & citations
transportation~\cite{aickelin_improved_2006, goel_truck_2012, gunther_combined_2010},
manufacturing~\cite{al-yakoob_mixed-integer_2007, al-yakoob_column_2008, alfares_simulation_2007},
retail~\cite{chapados_retail_2011, nissen_automatic_2010}, and
military~\cite{horn_scheduling_2007, laguna_modeling_2005}.
Of special interest to a clinician on-call scheduling
problem are the approaches to scheduling nurses, who often work in shifts. In the
nurse scheduling problem, the goal is to find an optimal assignment of nurses to
shifts that satisfies all of the hard constraints, such as hospital regulations,
and as many soft constraints as possible, which may include nurse preferences. A
% JK: Is this specific to the nurse scheduling problem?
%     This sounds like the more general definition of the scheduling problem.
%     Can you define hard and soft constraints here and give
%     one example of a hard constraint and one example of a soft constraint.
wide variety of approaches, including exact and heuristic approaches, have been
used to solve the nurse scheduling problem:
integer linear programming~\cite{azaiez_0-1_2005, trilling_nurse_2006, widyastiti_nurses_2016},
network flows~\cite{el_adoly_new_2018},
genetic algorithms~\cite{aickelin_exploiting_2000, jan_evolutionary_2000, kawanaka_genetic_2001},
simulated annealing~\cite{jaszkiewicz_metaheuristic_1997}, and
artificial intelligence~\cite{abdennadher_nurse_nodate, li_hybrid_2003}.
A comprehensive
literature review of these and other methods applied to nurse scheduling is
presented in~\cite{burke_state_2004}.
% JK: rostering = scheduling? I think okay to re-use "scheduling" unless its getting extremely repetitive.
%     but even then I think SM would agree clarity > flow/repetition.

%An extensive literature review of these and other methods is presented by [??].
%We will briefly summarize the main ideas of some of these approaches. \\ SM -
%don't need to introduce that you will do this for this type of paper I think -
%would do in a thesis chapter though.

Many of the approaches to nurse scheduling were designed to satisfy the
requirements of a specific hospital department, which results in a large number of
variables and constraints to be incorporated into the problem formulation. While
these department-specific approaches allow end-users to find precise schedules
that satisfy the needs of that department and the preferences of nurses and
clinicians in that department, they are difficult to adapt to other
departments. % explain why hard to
%adapt?... what makes their generalizablility/adaptability limited?
% JK: I'm less bothered by this lack of explaination, since I know its
%     hard to explain without getting into the nitty-gritty.
Moreover, the large number of variables and constraints also leads to
computational complexity issues~\cite{goos_complexity_1996}, especially when
% JK: I think good to clarify that these complextiy issues would imply that
%     the problem is essentially impossible to solve in a limited amount of time.
trying to find the most optimal solution. In this paper, we tackle a version
of the nurse scheduling problem arising from a case study of one clinical
division, providing two different services (general infectious
disease (ID) consults; and HIV consults) at St.\ Michael's Hospital in
Toronto, Canada. Our goals are to (1) present a integer linear programming
(ILP) formulation for the scheduling problem, and
% JK: the nurse scheduling problem?
%     I think we need to define the "scheduling problem"
%     and how / if it differs from other scheduling problem.
describe the flexibility of this formulation for solving similar problems;
(2) compare the performance of an ILP-based scheduling tool to the
results of a manual approach; and (3) analyze the robustness of this approach
in difficult instances of the problem.
% JK: what do you mean by different instances?  Can you be more specific?
%     The results explore:
%     (1) runtime vs number of requests
%     (2) runtime vs number of blocks
%     (3) satisfaction of hard and soft constraints.
% present a simple formulation for the problem developed and tested at the
%hospital after switching from a manual approach to scheduling; and (2) analyze
%the performance of integer linear programming in solving difficult instances of
%the problem and compare the results with those of the manual approach; and (3)
%describe the adaptability of the formulation as a basic framework for solving
%similar problems in other departments. \\ %make #3 an objective

We begin by describing the details of the problem in Section~\ref{sec:problem}, and presenting
our ILP formulation in Section~\ref{sec:methods}. Next, we compare the results
of our formulation to manually-created schedules, and evaluate the performance
of the algorithm on simulated data in Section~\ref{sec:results}. Finally, we
discuss and interpret the results in Section~\ref{sec:discussion}. % list the
%main contents/elements of the paper.
