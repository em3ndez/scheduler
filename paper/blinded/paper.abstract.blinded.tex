Scheduling of personnel in a hospital environment is vital to improving the
service provided to patients and balancing the workload assigned to clinicians.
Many approaches have been tried and successfully applied to generate efficient
schedules in such settings. However, due to the computational complexity of the
scheduling problem in general, most approaches resort to heuristics to find a
non-optimal solution in a reasonable amount of time. We designed an integer
linear programming formulation to find an optimal schedule in a clinical
division of a hospital. Our formulation mitigates issues related to
computational complexity by minimizing the set of constraints, yet retains
sufficient flexibility so that it can be adapted to a variety of clinical
divisions. \\

We then conducted a case study for our approach using data from the Infectious
Diseases division at our candidate hospital. We analyzed and
compared the results of our approach to manually-created schedules at the
hospital, and found improved adherence to departmental constraints and clinician
preferences. We used simulated data to examine the sensitivity of the runtime of
our linear program for various parameters and observed reassuring results,
signifying the practicality and generalizability of our approach in different
real-world scenarios.
