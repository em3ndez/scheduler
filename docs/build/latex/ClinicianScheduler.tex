%% Generated by Sphinx.
\def\sphinxdocclass{report}
\documentclass[letterpaper,10pt,english]{sphinxmanual}
\ifdefined\pdfpxdimen
   \let\sphinxpxdimen\pdfpxdimen\else\newdimen\sphinxpxdimen
\fi \sphinxpxdimen=.75bp\relax

\PassOptionsToPackage{warn}{textcomp}
\usepackage[utf8]{inputenc}
\ifdefined\DeclareUnicodeCharacter
 \ifdefined\DeclareUnicodeCharacterAsOptional
  \DeclareUnicodeCharacter{"00A0}{\nobreakspace}
  \DeclareUnicodeCharacter{"2500}{\sphinxunichar{2500}}
  \DeclareUnicodeCharacter{"2502}{\sphinxunichar{2502}}
  \DeclareUnicodeCharacter{"2514}{\sphinxunichar{2514}}
  \DeclareUnicodeCharacter{"251C}{\sphinxunichar{251C}}
  \DeclareUnicodeCharacter{"2572}{\textbackslash}
 \else
  \DeclareUnicodeCharacter{00A0}{\nobreakspace}
  \DeclareUnicodeCharacter{2500}{\sphinxunichar{2500}}
  \DeclareUnicodeCharacter{2502}{\sphinxunichar{2502}}
  \DeclareUnicodeCharacter{2514}{\sphinxunichar{2514}}
  \DeclareUnicodeCharacter{251C}{\sphinxunichar{251C}}
  \DeclareUnicodeCharacter{2572}{\textbackslash}
 \fi
\fi
\usepackage{cmap}
\usepackage[T1]{fontenc}
\usepackage{amsmath,amssymb,amstext}
\usepackage{babel}
\usepackage{times}
\usepackage[Bjarne]{fncychap}
\usepackage{sphinx}

\usepackage{geometry}

% Include hyperref last.
\usepackage{hyperref}
% Fix anchor placement for figures with captions.
\usepackage{hypcap}% it must be loaded after hyperref.
% Set up styles of URL: it should be placed after hyperref.
\urlstyle{same}

\addto\captionsenglish{\renewcommand{\figurename}{Fig.}}
\addto\captionsenglish{\renewcommand{\tablename}{Table}}
\addto\captionsenglish{\renewcommand{\literalblockname}{Listing}}

\addto\captionsenglish{\renewcommand{\literalblockcontinuedname}{continued from previous page}}
\addto\captionsenglish{\renewcommand{\literalblockcontinuesname}{continues on next page}}

\addto\extrasenglish{\def\pageautorefname{page}}





\title{Clinician Scheduler Documentation}
\date{Jul 09, 2019}
\release{2.0}
\author{David Landsman}
\newcommand{\sphinxlogo}{\vbox{}}
\renewcommand{\releasename}{Release}
\makeindex

\begin{document}

\maketitle
\sphinxtableofcontents
\phantomsection\label{\detokenize{manual::doc}}


\begin{sphinxShadowBox}
\sphinxstyletopictitle{Table of Contents}
\begin{itemize}
\item {} 
\phantomsection\label{\detokenize{manual:id4}}{\hyperref[\detokenize{manual:clinician-configuration}]{\sphinxcrossref{Clinician Configuration}}}
\begin{itemize}
\item {} 
\phantomsection\label{\detokenize{manual:id5}}{\hyperref[\detokenize{manual:creating-a-new-configuration-file}]{\sphinxcrossref{Creating a new configuration file}}}

\item {} 
\phantomsection\label{\detokenize{manual:id6}}{\hyperref[\detokenize{manual:saving-the-configuration-file}]{\sphinxcrossref{Saving the configuration file}}}

\item {} 
\phantomsection\label{\detokenize{manual:id7}}{\hyperref[\detokenize{manual:loading-a-configuration-file}]{\sphinxcrossref{Loading a configuration file}}}

\item {} 
\phantomsection\label{\detokenize{manual:id8}}{\hyperref[\detokenize{manual:adding-a-new-clinician}]{\sphinxcrossref{Adding a new clinician}}}

\item {} 
\phantomsection\label{\detokenize{manual:id9}}{\hyperref[\detokenize{manual:deleting-an-existing-clinician}]{\sphinxcrossref{Deleting an existing clinician}}}

\item {} 
\phantomsection\label{\detokenize{manual:id10}}{\hyperref[\detokenize{manual:editing-an-existing-clinician}]{\sphinxcrossref{Editing an existing clinician}}}

\end{itemize}

\item {} 
\phantomsection\label{\detokenize{manual:id11}}{\hyperref[\detokenize{manual:requests-holiday-weekends}]{\sphinxcrossref{Requests \& Holiday Weekends}}}
\begin{itemize}
\item {} 
\phantomsection\label{\detokenize{manual:id12}}{\hyperref[\detokenize{manual:clinician-requests-excel-file}]{\sphinxcrossref{Clinician Requests Excel File}}}

\item {} 
\phantomsection\label{\detokenize{manual:id13}}{\hyperref[\detokenize{manual:holiday-weekends-excel-file}]{\sphinxcrossref{Holiday Weekends Excel File}}}

\end{itemize}

\item {} 
\phantomsection\label{\detokenize{manual:id14}}{\hyperref[\detokenize{manual:scheduling}]{\sphinxcrossref{Scheduling}}}
\begin{itemize}
\item {} 
\phantomsection\label{\detokenize{manual:id15}}{\hyperref[\detokenize{manual:generating-a-schedule}]{\sphinxcrossref{Generating a schedule}}}

\item {} 
\phantomsection\label{\detokenize{manual:id16}}{\hyperref[\detokenize{manual:exporting-a-schedule}]{\sphinxcrossref{Exporting a schedule}}}

\end{itemize}

\item {} 
\phantomsection\label{\detokenize{manual:id17}}{\hyperref[\detokenize{manual:sample-output}]{\sphinxcrossref{Sample Output}}}
\begin{itemize}
\item {} 
\phantomsection\label{\detokenize{manual:id18}}{\hyperref[\detokenize{manual:yearly-excel-format}]{\sphinxcrossref{Yearly Excel Format}}}

\item {} 
\phantomsection\label{\detokenize{manual:id19}}{\hyperref[\detokenize{manual:monthly-excel-format}]{\sphinxcrossref{Monthly Excel Format}}}

\end{itemize}

\end{itemize}
\end{sphinxShadowBox}


\chapter{Clinician Configuration}
\label{\detokenize{manual:clinician-configuration}}\label{\detokenize{manual:id1}}
The clinician configuration specifies which clinicians are available,
which divisions they are covering, and how many weeks they should cover
in each of their divisions.


\section{Creating a new configuration file}
\label{\detokenize{manual:creating-a-new-configuration-file}}
By default, you will get a blank configuration file when you launch
the program. If you would like to discard the changes you have made and
start a configuration file from scratch, simply click on \sphinxtitleref{New Config}.

\begin{figure}[H]
\centering
\sphinxhref{\_static/images/configuration/new\_config.png}{\sphinxincludegraphics{{new_config}.png}}\end{figure}

\begin{sphinxadmonition}{warning}{Warning:}
Unsaved changes to a configuration file will be discarded upon clicking
on \sphinxtitleref{New Config}.
\end{sphinxadmonition}


\section{Saving the configuration file}
\label{\detokenize{manual:saving-the-configuration-file}}
When you are ready to save the configuration you created, click on
\sphinxtitleref{Save Config} and choose a place to save your file. Make note of the name
and directory of the file so you could load it in future runs.

\begin{figure}[H]
\centering
\sphinxhref{\_static/images/configuration/save\_config.png}{\sphinxincludegraphics{{save_config}.png}}\end{figure}


\section{Loading a configuration file}
\label{\detokenize{manual:loading-a-configuration-file}}
If you would like to open a previously created configuration file, simply
click on \sphinxtitleref{Open Config}, navigate to the location of the file,
and select it.

\begin{figure}[H]
\centering
\sphinxhref{\_static/images/configuration/load\_config.png}{\sphinxincludegraphics{{load_config}.png}}\end{figure}


\section{Adding a new clinician}
\label{\detokenize{manual:adding-a-new-clinician}}\label{\detokenize{manual:id2}}
1. From the configuration tab, click \sphinxtitleref{New Clinician}. You should see a
form for supplying clinician details.

\begin{figure}[H]
\centering
\sphinxhref{\_static/images/configuration/add\_clinician/step1\_new\_clinician.png}{\sphinxincludegraphics{{step1_new_clinician}.png}}\end{figure}
\begin{enumerate}
\setcounter{enumi}{1}
\item {} 
Fill out the name, email (optional), and divisions that the clinician
will be covering. To add a division you can click on \sphinxtitleref{Add} and a new row
will be added to the table which you can fill out. You can set the minimum
and maximum number of blocks that a clinician can work in a given division.

\end{enumerate}

\begin{sphinxadmonition}{note}{Note:}
A single block corresponds to two weeks.
\end{sphinxadmonition}

\begin{figure}[H]
\centering
\sphinxhref{\_static/images/configuration/add\_clinician/step2\_add\_division.png}{\sphinxincludegraphics{{step2_add_division}.png}}\end{figure}
\begin{enumerate}
\setcounter{enumi}{2}
\item {} 
To delete a row from the table, select the row and then click \sphinxtitleref{Remove}.

\end{enumerate}

\begin{figure}[H]
\centering
\sphinxhref{\_static/images/configuration/add\_clinician/step3\_remove\_division.png}{\sphinxincludegraphics{{step3_remove_division}.png}}\end{figure}
\begin{enumerate}
\setcounter{enumi}{3}
\item {} 
When you are finished entering the data for the clinician, click \sphinxtitleref{Ok}.
You should now see a new entry in the main table for that clinician.

\end{enumerate}

\begin{figure}[H]
\centering
\sphinxhref{\_static/images/configuration/add\_clinician/step4\_add\_clinician.png}{\sphinxincludegraphics{{step4_add_clinician}.png}}\end{figure}


\section{Deleting an existing clinician}
\label{\detokenize{manual:deleting-an-existing-clinician}}
To delete an existing clinician, simply select a row corresponding
to the clinician in the table and click on \sphinxtitleref{Delete Clinician}.

\begin{figure}[H]
\centering
\sphinxhref{\_static/images/configuration/delete\_clinician.png}{\sphinxincludegraphics{{delete_clinician}.png}}\end{figure}


\section{Editing an existing clinician}
\label{\detokenize{manual:editing-an-existing-clinician}}
To edit the information of a clinician, select a row corresponding
to the clinician in the table and click on \sphinxtitleref{Edit Clinician}. You should
see a dialog window where you can change the information. For more
details on how to enter data in the edit dialog, see {\hyperref[\detokenize{manual:adding-a-new-clinician}]{\sphinxcrossref{\DUrole{std,std-ref}{Adding a new clinician}}}}.

\begin{figure}[H]
\centering
\sphinxhref{\_static/images/configuration/edit\_clinician.png}{\sphinxincludegraphics{{edit_clinician}.png}}\end{figure}


\chapter{Requests \& Holiday Weekends}
\label{\detokenize{manual:requests-holiday-weekends}}\label{\detokenize{manual:requests-and-holidays}}
Clinician requests for time-off and the dates of holiday weekends are
supplied using a pair of excel files.


\section{Clinician Requests Excel File}
\label{\detokenize{manual:clinician-requests-excel-file}}
The requests file should have a single sheet per clinician that is in the
configuration file you created in {\hyperref[\detokenize{manual:clinician-configuration}]{\sphinxcrossref{\DUrole{std,std-ref}{Clinician Configuration}}}}. Inside
each sheet, each request is entered on a separate row, containing the start
and end date (inclusive) of the request.

\begin{sphinxadmonition}{warning}{Warning:}
Make sure each value in the Excel file is formatted as a date!
\end{sphinxadmonition}

\begin{figure}[H]
\centering
\sphinxhref{\_static/images/requests\_holidays/excel\_requests.png}{\sphinxincludegraphics{{excel_requests}.png}}\end{figure}

\begin{sphinxadmonition}{note}{Note:}
To enter a request for a single day, use the same value for both start
and end date.
\end{sphinxadmonition}


\section{Holiday Weekends Excel File}
\label{\detokenize{manual:holiday-weekends-excel-file}}
The holiday weekends file should have a single sheet with the dates of the
holiday weekends, one in each row.

\begin{sphinxadmonition}{warning}{Warning:}
Make sure each value in the Excel file is formatted as a date!
\end{sphinxadmonition}

\begin{figure}[H]
\centering
\sphinxhref{\_static/images/requests\_holidays/excel\_holidays.png}{\sphinxincludegraphics{{excel_holidays}.png}}\end{figure}


\chapter{Scheduling}
\label{\detokenize{manual:scheduling}}

\section{Generating a schedule}
\label{\detokenize{manual:generating-a-schedule}}\label{\detokenize{manual:id3}}
Once you have created a configuration file, you can switch over to the
\sphinxtitleref{Scheduler} tab of the application in order to generate a schedule.
\begin{enumerate}
\item {} 
Load the configuration, requests, and holidays files that you created
earlier in {\hyperref[\detokenize{manual:clinician-configuration}]{\sphinxcrossref{\DUrole{std,std-ref}{Clinician Configuration}}}} and {\hyperref[\detokenize{manual:requests-and-holidays}]{\sphinxcrossref{\DUrole{std,std-ref}{Requests \& Holiday Weekends}}}}.

\end{enumerate}

\begin{sphinxadmonition}{note}{Note:}
You can be sure that everything loaded correctly by checking the output
on the right.
\end{sphinxadmonition}

\begin{figure}[H]
\centering
\sphinxhref{\_static/images/scheduling/generate\_schedule/step1\_load.png}{\sphinxincludegraphics{{step1_load}.png}}\end{figure}
\begin{enumerate}
\setcounter{enumi}{1}
\item {} 
Set the starting calendar year for the schedule, as well as the number
of 2-week blocks that you want the schedule to cover. By default, the calendar
year is the upcoming year, and the number of blocks is 26, to cover a full
year.

\end{enumerate}

\begin{figure}[H]
\centering
\sphinxhref{\_static/images/scheduling/generate\_schedule/step2\_year\_blocks.png}{\sphinxincludegraphics{{step2_year_blocks}.png}}\end{figure}
\begin{enumerate}
\setcounter{enumi}{2}
\item {} 
\sphinxstylestrong{(Optional)} Check the \sphinxtitleref{Shuffle?} checkbox if you would like the scheduler
to generate a slightly different schedule each time you click on \sphinxtitleref{Generate}.

\end{enumerate}

\begin{figure}[H]
\centering
\sphinxhref{\_static/images/scheduling/generate\_schedule/step3\_optional\_shuffle.png}{\sphinxincludegraphics{{step3_optional_shuffle}.png}}\end{figure}
\begin{enumerate}
\setcounter{enumi}{3}
\item {} 
\sphinxstylestrong{(Optional)} Check the \sphinxtitleref{Verbose Output} checkbox if you would like
to see detailed output on the right after the scheduler finishes generating
the schedule.

\end{enumerate}

\begin{figure}[H]
\centering
\sphinxhref{\_static/images/scheduling/generate\_schedule/step4\_optional\_verbose.png}{\sphinxincludegraphics{{step4_optional_verbose}.png}}\end{figure}
\begin{enumerate}
\setcounter{enumi}{4}
\item {} 
Click on \sphinxtitleref{Generate} to generate a schedule.

\end{enumerate}

\begin{sphinxadmonition}{warning}{Warning:}
Depending on the amount of clinicians and requests provided, it may take
some time to find an optimal schedule.
\end{sphinxadmonition}

\begin{figure}[H]
\centering
\sphinxhref{\_static/images/scheduling/generate\_schedule/step5\_generate.png}{\sphinxincludegraphics{{step5_generate}.png}}\end{figure}

\begin{sphinxadmonition}{warning}{Warning:}
It is possible that the scheduler will not be able to come up with a
schedule that satisfies your constraints. This can be a result of either
of the following reasons:
\begin{itemize}
\item {} 
There are not enough clinicians to distribute evenly throughout the year

\item {} 
The minimum and maximum number of blocks of clinicians are too restrictive

\end{itemize}

Try changing the configuration file by adding new clinicians, or changing
the min/max blocks of different clinicians to allow for more flexibility.

See {\hyperref[\detokenize{manual:clinician-configuration}]{\sphinxcrossref{\DUrole{std,std-ref}{Clinician Configuration}}}} for more information on modifying the configuration file.
\end{sphinxadmonition}


\section{Exporting a schedule}
\label{\detokenize{manual:exporting-a-schedule}}
If you are satisfied with the generated schedule, you can choose to export
it as an Excel file. There are two Excel format options: \sphinxtitleref{Yearly Export} and
\sphinxtitleref{Monthly Export}.

Selecting the \sphinxtitleref{Yearly Export} option will generate an excel file with a single
sheet, displaying the clinicians that are covering a particular division
for a given week or weekend. It is very similar to the table output in
the application itself.

\begin{figure}[H]
\centering
\sphinxhref{\_static/images/scheduling/export\_schedule/export\_yearly.png}{\sphinxincludegraphics{{export_yearly}.png}}\end{figure}

Selecting the \sphinxtitleref{Monthly Export} option will generate a more detailed breakdown
of the schedule, with a separate sheet for every month, detailing which
clinician covers which division on which day.

\begin{figure}[H]
\centering
\sphinxhref{\_static/images/scheduling/export\_schedule/export\_monthly.png}{\sphinxincludegraphics{{export_monthly}.png}}\end{figure}


\chapter{Sample Output}
\label{\detokenize{manual:sample-output}}

\section{Yearly Excel Format}
\label{\detokenize{manual:yearly-excel-format}}
\begin{figure}[H]
\centering
\sphinxhref{\_static/images/scheduling/sample\_yearly.png}{\sphinxincludegraphics{{sample_yearly}.png}}\end{figure}


\section{Monthly Excel Format}
\label{\detokenize{manual:monthly-excel-format}}
\begin{figure}[H]
\centering
\sphinxhref{\_static/images/scheduling/sample\_monthly.png}{\sphinxincludegraphics{{sample_monthly}.png}}\end{figure}



\renewcommand{\indexname}{Index}
\printindex
\end{document}