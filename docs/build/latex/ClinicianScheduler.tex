%% Generated by Sphinx.
\def\sphinxdocclass{report}
\documentclass[letterpaper,10pt,english]{sphinxmanual}
\ifdefined\pdfpxdimen
   \let\sphinxpxdimen\pdfpxdimen\else\newdimen\sphinxpxdimen
\fi \sphinxpxdimen=.75bp\relax

\PassOptionsToPackage{warn}{textcomp}
\usepackage[utf8]{inputenc}
\ifdefined\DeclareUnicodeCharacter
 \ifdefined\DeclareUnicodeCharacterAsOptional
  \DeclareUnicodeCharacter{"00A0}{\nobreakspace}
  \DeclareUnicodeCharacter{"2500}{\sphinxunichar{2500}}
  \DeclareUnicodeCharacter{"2502}{\sphinxunichar{2502}}
  \DeclareUnicodeCharacter{"2514}{\sphinxunichar{2514}}
  \DeclareUnicodeCharacter{"251C}{\sphinxunichar{251C}}
  \DeclareUnicodeCharacter{"2572}{\textbackslash}
 \else
  \DeclareUnicodeCharacter{00A0}{\nobreakspace}
  \DeclareUnicodeCharacter{2500}{\sphinxunichar{2500}}
  \DeclareUnicodeCharacter{2502}{\sphinxunichar{2502}}
  \DeclareUnicodeCharacter{2514}{\sphinxunichar{2514}}
  \DeclareUnicodeCharacter{251C}{\sphinxunichar{251C}}
  \DeclareUnicodeCharacter{2572}{\textbackslash}
 \fi
\fi
\usepackage{cmap}
\usepackage[T1]{fontenc}
\usepackage{amsmath,amssymb,amstext}
\usepackage{babel}
\usepackage{times}
\usepackage[Bjarne]{fncychap}
\usepackage{sphinx}

\usepackage{geometry}

% Include hyperref last.
\usepackage{hyperref}
% Fix anchor placement for figures with captions.
\usepackage{hypcap}% it must be loaded after hyperref.
% Set up styles of URL: it should be placed after hyperref.
\urlstyle{same}

\addto\captionsenglish{\renewcommand{\figurename}{Fig.}}
\addto\captionsenglish{\renewcommand{\tablename}{Table}}
\addto\captionsenglish{\renewcommand{\literalblockname}{Listing}}

\addto\captionsenglish{\renewcommand{\literalblockcontinuedname}{continued from previous page}}
\addto\captionsenglish{\renewcommand{\literalblockcontinuesname}{continues on next page}}

\addto\extrasenglish{\def\pageautorefname{page}}





\title{Clinician Scheduler Documentation}
\date{Jan 24, 2019}
\release{1.0}
\author{David Landsman}
\newcommand{\sphinxlogo}{\vbox{}}
\renewcommand{\releasename}{Release}
\makeindex

\begin{document}

\maketitle
\sphinxtableofcontents
\phantomsection\label{\detokenize{index::doc}}


\begin{sphinxShadowBox}
\sphinxstyletopictitle{Contents:}
\begin{itemize}
\item {} 
\phantomsection\label{\detokenize{index:id2}}{\hyperref[\detokenize{index:setup}]{\sphinxcrossref{Setup}}}
\begin{itemize}
\item {} 
\phantomsection\label{\detokenize{index:id3}}{\hyperref[\detokenize{index:generating-google-api-credentials}]{\sphinxcrossref{Generating Google API credentials}}}

\end{itemize}

\item {} 
\phantomsection\label{\detokenize{index:id4}}{\hyperref[\detokenize{index:usage}]{\sphinxcrossref{Usage}}}
\begin{itemize}
\item {} 
\phantomsection\label{\detokenize{index:id5}}{\hyperref[\detokenize{index:clinician-configuration}]{\sphinxcrossref{Clinician Configuration}}}
\begin{itemize}
\item {} 
\phantomsection\label{\detokenize{index:id6}}{\hyperref[\detokenize{index:adding-a-new-clinician}]{\sphinxcrossref{Adding a new clinician}}}

\item {} 
\phantomsection\label{\detokenize{index:id7}}{\hyperref[\detokenize{index:deleting-an-existing-clinician}]{\sphinxcrossref{Deleting an existing clinician}}}

\item {} 
\phantomsection\label{\detokenize{index:id8}}{\hyperref[\detokenize{index:editing-an-existing-clinician}]{\sphinxcrossref{Editing an existing clinician}}}

\item {} 
\phantomsection\label{\detokenize{index:id9}}{\hyperref[\detokenize{index:saving-the-configuration-to-a-file}]{\sphinxcrossref{Saving the configuration to a file}}}

\item {} 
\phantomsection\label{\detokenize{index:id10}}{\hyperref[\detokenize{index:loading-a-configuration-file}]{\sphinxcrossref{Loading a configuration file}}}

\end{itemize}

\item {} 
\phantomsection\label{\detokenize{index:id11}}{\hyperref[\detokenize{index:google-calendar-configuration}]{\sphinxcrossref{Google Calendar Configuration}}}
\begin{itemize}
\item {} 
\phantomsection\label{\detokenize{index:id12}}{\hyperref[\detokenize{index:creating-a-calendar}]{\sphinxcrossref{Creating a calendar}}}

\item {} 
\phantomsection\label{\detokenize{index:id13}}{\hyperref[\detokenize{index:adding-holiday-events}]{\sphinxcrossref{Adding holiday events}}}

\item {} 
\phantomsection\label{\detokenize{index:id14}}{\hyperref[\detokenize{index:adding-clinician-requests}]{\sphinxcrossref{Adding clinician requests}}}

\end{itemize}

\item {} 
\phantomsection\label{\detokenize{index:id15}}{\hyperref[\detokenize{index:scheduling}]{\sphinxcrossref{Scheduling}}}
\begin{itemize}
\item {} 
\phantomsection\label{\detokenize{index:id16}}{\hyperref[\detokenize{index:id1}]{\sphinxcrossref{Loading a configuration file}}}

\item {} 
\phantomsection\label{\detokenize{index:id17}}{\hyperref[\detokenize{index:generating-a-schedule}]{\sphinxcrossref{Generating a schedule}}}

\item {} 
\phantomsection\label{\detokenize{index:id18}}{\hyperref[\detokenize{index:exporting-a-schedule}]{\sphinxcrossref{Exporting a schedule}}}

\item {} 
\phantomsection\label{\detokenize{index:id19}}{\hyperref[\detokenize{index:publishing-a-schedule-to-google-calendar}]{\sphinxcrossref{Publishing a schedule to Google Calendar}}}

\end{itemize}

\end{itemize}

\end{itemize}
\end{sphinxShadowBox}

The clinician scheduler allows you to automatically generate and publish
schedules that satisfy common constraints in an on-call system, while taking
into account the preferences of clinicians.


\chapter{Setup}
\label{\detokenize{index:setup}}
Download and unzip the compressed file to an appropriate folder.


\section{Generating Google API credentials}
\label{\detokenize{index:generating-google-api-credentials}}
\begin{sphinxadmonition}{note}{Note:}
This process should only be done the very first time you are running
the program. If you have previously generated a credential file using
this process you should be able to re-use it. Just make sure that the
credential file is placed in the same folder as the executable (\sphinxcode{\sphinxupquote{scheduler.exe}}).
\end{sphinxadmonition}

The application uses Google calendar for retrieving clinician
time-off requests and long weekend information, as well as uploading the
generated schedule to the calendar. These operations require the use
of Google API credentials, which can be generated as follows.
\begin{enumerate}
\item {} 
Sign into \sphinxurl{https://console.developers.google.com}.

\item {} 
Click on the project selector at the top left of the page.

\end{enumerate}

\begin{figure}[htbp]
\centering
\sphinxhref{\_static/images/google\_credentials/step2\_select\_a\_project.png}{\sphinxincludegraphics{{step2_select_a_project}.png}}\end{figure}

\begin{DUlineblock}{0em}
\item[] 
\end{DUlineblock}
\begin{enumerate}
\setcounter{enumi}{2}
\item {} 
Click on \sphinxtitleref{New Project}.

\end{enumerate}

\begin{figure}[htbp]
\centering
\sphinxhref{\_static/images/google\_credentials/step3\_new\_project.png}{\sphinxincludegraphics{{step3_new_project}.png}}\end{figure}

\begin{DUlineblock}{0em}
\item[] 
\end{DUlineblock}
\begin{enumerate}
\setcounter{enumi}{3}
\item {} 
Enter “Clinician Scheduler” as the \sphinxtitleref{Project Name} and click \sphinxtitleref{Create}.

\end{enumerate}

\begin{figure}[htbp]
\centering
\sphinxhref{\_static/images/google\_credentials/step4\_create\_project.png}{\sphinxincludegraphics{{step4_create_project}.png}}\end{figure}

\begin{DUlineblock}{0em}
\item[] 
\end{DUlineblock}

5. Now you should see the dashboard for the Clinician Scheduler project.
You will need to enable the Google calendar API. Click on \sphinxtitleref{Enable APIs and Services}.

\begin{figure}[htbp]
\centering
\sphinxhref{\_static/images/google\_credentials/step5\_enable\_apis\_and\_services.png}{\sphinxincludegraphics{{step5_enable_apis_and_services}.png}}\end{figure}

\begin{DUlineblock}{0em}
\item[] 
\end{DUlineblock}
\begin{enumerate}
\setcounter{enumi}{5}
\item {} 
Search for “Google calendar API” using the search bar, and select it.

\end{enumerate}

\begin{figure}[htbp]
\centering
\sphinxhref{\_static/images/google\_credentials/step6\_search\_calendar\_api.png}{\sphinxincludegraphics{{step6_search_calendar_api}.png}}\end{figure}

\begin{DUlineblock}{0em}
\item[] 
\end{DUlineblock}
\begin{enumerate}
\setcounter{enumi}{6}
\item {} 
Click \sphinxtitleref{Enable}.

\end{enumerate}

\begin{figure}[htbp]
\centering
\sphinxhref{\_static/images/google\_credentials/step7\_enable\_calendar\_api.png}{\sphinxincludegraphics{{step7_enable_calendar_api}.png}}\end{figure}

\begin{DUlineblock}{0em}
\item[] 
\end{DUlineblock}

8. Now you should see the overview page for the Google calendar API.
To generate the credentials, click \sphinxtitleref{Create credentials}.

\begin{figure}[htbp]
\centering
\sphinxhref{\_static/images/google\_credentials/step8\_create\_credentials.png}{\sphinxincludegraphics{{step8_create_credentials}.png}}\end{figure}

\begin{DUlineblock}{0em}
\item[] 
\end{DUlineblock}

9. On the credentials form, choose “Google Calendar API” for \sphinxtitleref{Which API are you using?},
then “Other UI (e.g. Windows, CLI tool)” for \sphinxtitleref{Where will you be calling the API from?}
and “User data” for \sphinxtitleref{What data will you be accessing?}. Then click on \sphinxtitleref{What credentials do I need?}.

\begin{figure}[htbp]
\centering
\sphinxhref{\_static/images/google\_credentials/step9\_credentials\_form.png}{\sphinxincludegraphics{{step9_credentials_form}.png}}\end{figure}

\begin{DUlineblock}{0em}
\item[] 
\end{DUlineblock}
\begin{enumerate}
\setcounter{enumi}{9}
\item {} 
Enter “Client” for \sphinxtitleref{Name} and click on \sphinxtitleref{Create OAuth client ID}.

\end{enumerate}

\begin{figure}[htbp]
\centering
\sphinxhref{\_static/images/google\_credentials/step10\_oauth\_client\_id.png}{\sphinxincludegraphics{{step10_oauth_client_id}.png}}\end{figure}

\begin{DUlineblock}{0em}
\item[] 
\end{DUlineblock}

11. Choose the email address associated to your account for \sphinxtitleref{Email address}
and enter “Clinician Scheduler” for \sphinxtitleref{Product name shown to users}, then
click \sphinxtitleref{Continue}.

\begin{figure}[htbp]
\centering
\sphinxhref{\_static/images/google\_credentials/step11\_oauth\_consent\_screen.png}{\sphinxincludegraphics{{step11_oauth_consent_screen}.png}}\end{figure}

\begin{DUlineblock}{0em}
\item[] 
\end{DUlineblock}

12. Your credentials are now generated! Make sure to download and save them
in the same location that you unzipped the application, so that the
credential file and the executable file (\sphinxcode{\sphinxupquote{scheduler.exe}}) are in the same folder.

\begin{figure}[htbp]
\centering
\sphinxhref{\_static/images/google\_credentials/step12\_download\_credentials.png}{\sphinxincludegraphics{{step12_download_credentials}.png}}\end{figure}

\begin{DUlineblock}{0em}
\item[] 
\end{DUlineblock}

\begin{sphinxadmonition}{attention}{Attention:}
Make sure the credential file is saved as \sphinxcode{\sphinxupquote{credentials.json}} (rename it, if necessary),
or otherwise the application will not be able to recognize it!
\end{sphinxadmonition}


\chapter{Usage}
\label{\detokenize{index:usage}}

\section{Clinician Configuration}
\label{\detokenize{index:clinician-configuration}}
Before we can generate a schedule, we need to create a configuration file
that specifies which clinicians are available, and how many weeks each
clinician should fulfill.


\subsection{Adding a new clinician}
\label{\detokenize{index:adding-a-new-clinician}}
1. From the configuration tab, click \sphinxtitleref{New Clinician}. You should see a
form for supplying details.

\begin{figure}[htbp]
\centering
\sphinxhref{\_static/images/add\_clinician/step1\_new\_clinician.png}{\sphinxincludegraphics{{step1_new_clinician}.png}}\end{figure}

\begin{DUlineblock}{0em}
\item[] 
\end{DUlineblock}

2. Fill out the name, email (optional), and divisions that the clinician
will be covering. To add a division you can click on \sphinxtitleref{Add} and a new row
will be added to the table which you can fill out. You can set the minimum
and maximum number of blocks that a clinician can work in a given division.

\begin{sphinxadmonition}{note}{Note:}
A single block corresponds to two weeks.
\end{sphinxadmonition}

\begin{figure}[htbp]
\centering
\sphinxhref{\_static/images/add\_clinician/step2\_add\_division.png}{\sphinxincludegraphics{{step2_add_division}.png}}\end{figure}

\begin{DUlineblock}{0em}
\item[] 
\end{DUlineblock}
\begin{enumerate}
\setcounter{enumi}{2}
\item {} 
To delete a row from the table, select the row and then click \sphinxtitleref{Remove}.

\end{enumerate}

\begin{figure}[htbp]
\centering
\sphinxhref{\_static/images/add\_clinician/step3\_remove\_division.png}{\sphinxincludegraphics{{step3_remove_division}.png}}\end{figure}

\begin{DUlineblock}{0em}
\item[] 
\end{DUlineblock}

4. When you are finished entering the data for the clinician, click \sphinxtitleref{Ok}.
You should now see a new entry in the main table for that clinician.

\begin{figure}[htbp]
\centering
\sphinxhref{\_static/images/add\_clinician/step4\_add\_clinician.png}{\sphinxincludegraphics{{step4_add_clinician}.png}}\end{figure}

\begin{DUlineblock}{0em}
\item[] 
\end{DUlineblock}


\subsection{Deleting an existing clinician}
\label{\detokenize{index:deleting-an-existing-clinician}}

\subsection{Editing an existing clinician}
\label{\detokenize{index:editing-an-existing-clinician}}

\subsection{Saving the configuration to a file}
\label{\detokenize{index:saving-the-configuration-to-a-file}}

\subsection{Loading a configuration file}
\label{\detokenize{index:loading-a-configuration-file}}

\section{Google Calendar Configuration}
\label{\detokenize{index:google-calendar-configuration}}

\subsection{Creating a calendar}
\label{\detokenize{index:creating-a-calendar}}

\subsection{Adding holiday events}
\label{\detokenize{index:adding-holiday-events}}

\subsection{Adding clinician requests}
\label{\detokenize{index:adding-clinician-requests}}

\section{Scheduling}
\label{\detokenize{index:scheduling}}

\subsection{Loading a configuration file}
\label{\detokenize{index:id1}}

\subsection{Generating a schedule}
\label{\detokenize{index:generating-a-schedule}}

\subsection{Exporting a schedule}
\label{\detokenize{index:exporting-a-schedule}}

\subsection{Publishing a schedule to Google Calendar}
\label{\detokenize{index:publishing-a-schedule-to-google-calendar}}


\renewcommand{\indexname}{Index}
\printindex
\end{document}