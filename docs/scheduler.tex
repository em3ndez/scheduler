\documentclass[]{article}

\usepackage{amsmath}
\usepackage{amsthm}
\usepackage{amssymb}
\usepackage{mathtools}
\usepackage{hyperref}

\newcommand{\mc}{\mathcal}
\newcommand{\bb}{\mathbb}
\DeclarePairedDelimiter{\floor}{\lfloor}{\rfloor}
\DeclarePairedDelimiter{\ceil}{\lceil}{\rceil}

%opening
\title{Application of Network Flow and Linear Programming to Scheduling of Clinicians}
\author{David Landsman}

\begin{document}
\maketitle

\section{Problem} \label{problem}
A group of clinicians are working on-call at a clinic each day of the year, including weekends and holidays, at multiple divisions. When a clinician is assigned to work during the week, they work from 8 A.M. on Monday to 5 P.M. on Friday. During weekends, a clinician works the complement period of time, that is, Friday 5 P.M. to Monday 8 A.M. Each clinician can request not to work certain weeks and weekends of the year. We refer to this as the clinician's \textit{time-off}.

\subsection{Constraints} \label{constraints}
We formalize the constraints of the problem to be able to describe them mathematically later on.
	\begin{enumerate}
		\item Each division needs to have one and only one clinician that covers a every block
		\item Every weekend needs to have one and only one clinician that covers it
		\item For each division, each clinician can only work between the minimum and maximum number of weeks they are allowed (note: these may be different for different divisions)
		\item Each clinician should work a given number of long weekends. Roughly, this is the number of long weekends divided by the number of clinicians
		\item A clinician cannot work two consecutive blocks (either in the same division, or in different divisions)
		\item A clinician cannot work two consecutive weekends
	\end{enumerate}

\subsection{Objectives} \label{objectives}
Since it is not always possible to accommodate all time-off requests from all clinicians, we decided to use this metric as an objective instead of a constraint. 
	\begin{enumerate}
		\item Minimize the number of blocks assigned to a clinician that they requested as time-off
		\item Minimize the number of weekends assigned to a clinician that they requested as time-off
		\item Maximize the adjacency between blocks and weekends assigned for a given clinician. In particular, if a clinician is assigned to work a given block in a given division, we will prefer it if the clinician also works in the weekend between the weeks of the block.
	\end{enumerate}	

\section{Linear Programming Formulation}
The problem described in \ref{problem} corresponds to an Assignment Problem between clinicians and blocks/weekends. Such a problem can be presented as an Integer Linear Program (ILP).  \\ \\
We begin by defining the sets, constants, and indices that will be used in this formulation.
	\begin{align*}
		\mc{D}\coloneqq &\text{ the set of all divisions, indexed by } i \\
		\mc{C}\coloneqq &\text{ the set of all clinicians, indexed by } j \\
		\mc{B}\coloneqq &\text{ the set of all blocks, indexed by } k \\
		\mc{W}\coloneqq &\text{ the set of all weekends, indexed by } l \\
		\mc{W} \supset \mc{L}\coloneqq &\text{ the set of all long weekends} \\
		\mc{B} \supset \mc{S}_j\coloneqq &\text{ the set of blocks clinician } j \text{ requested off} \\
		\mc{W} \supset \mc{T}_j\coloneqq &\text{ the set of weekends clinician } j \text{ requested off} \\
		m^i_j \coloneqq &\text{ the minimum number of blocks clinician } j \text{ should work in division } i \\
		M^i_j \coloneqq &\text{ the maximum number of blocks clinician } j \text{ should work in division } i
	\end{align*}
We also define the set of 0-1 variables $X^i_{j,k}$ and $Y_{j,l}$ that we will solve for in the ILP.
	\begin{alignat}{2}
		&X^i_{j,k} \in \{0, 1\}: &&\text{ clinician } j \text{ covers block } k \text{ for division } i \\
		&Y_{j, l} \in \{0, 1\}: &&\text{ clinician } j \text{ covers weekend } l
	\end{alignat}
Now we can write the constraints in \ref{constraints} mathematically.
	\begin{enumerate}
		\item 
			\begin{equation}
				\sum_j X^i_{j, k} = 1 \text{ for each } i, k
			\end{equation}
		\item
			\begin{equation}
				\sum_j Y_{j, l} = 1 \text{ for each } l
			\end{equation}
		\item 
			\begin{equation}
				m^i_j \leq \sum_{k} X^i_{j, k} \leq M^i_j \text{ for each } i, j
			\end{equation}
		\item
			\begin{equation}
				\floor*{\frac{|\mc{L}|}{|\mc{C}|}} \leq \sum_{l \in \mc{L}} Y_{j, l} \leq \ceil*{\frac{|\mc{L}|}{|\mc{C}|}}
			\end{equation}
		\item
			\begin{equation}
				\sum_i \left(X^i_{j, k} + X^i_{j, k+1}\right) \leq 1 \text{ for each } j,k \text{ where } k \leq |\mc{B}| - 1
			\end{equation}
		\item
			\begin{equation}
				Y_{j, l} + Y_{j, l+1} \leq 1 \text{ for each } j, l \text{ where } l \leq |\mc{W}| - 1
			\end{equation}
	\end{enumerate}
Note that in ILP we can only have a single objective function. Since we have multiple objectives in \ref{objectives}, we will need to combine them in an appropriate manner. Assuming each objective can be summarized as a linear function of the variables defined above, say $Obj_n\left(X^i_{j, k}, Y_{j, l}\right)$, $1 \leq n \leq N$, then we shall define the overall objective function as a linear combination of $Obj_n$'s:
	\begin{equation}
		Obj\left(X^i_{j, k}, Y_{j, l}\right) \coloneqq \frac{Obj_1 + \ldots + Obj_N}{N} 
	\end{equation}
This ensures that all objectives have an equal weight of $\frac{1}{N}$. It is also necessary to choose whether we minimize or maximize $Obj$, so we will need to convert the minimization objectives. In particular, we can define the following objectives:
	\begin{enumerate}
		\item We will maximize the number of blocks assigned to a clinician that are \textit{not} requested for time-off:
			\begin{equation}
				Obj_1\left(X^i_{j, k}, Y_{j, l}\right) \coloneqq \sum_i \sum_j \sum_k
					\begin{cases}
					X^i_{j, k} &\text{ if } k \in \mc{S}_j \\
					-X^i_{j, k} &\text{ otherwise }
					\end{cases}
			\end{equation}
		\item We will maximize the number of weekends assigned to a clinician that are \textit{not} requested for time-off
			\begin{equation}
				Obj_2\left(X^i_{j, k}, Y_{j, l}\right) \coloneqq \sum_j \sum_l 
					\begin{cases}
					Y_{j, l} &\text{ if } l \in \mc{T}_j \\
					-Y_{j, l} &\text{ otherwise }
					\end{cases}
			\end{equation}
		\item We want to maximize the number of adjacent blocks and weekends. Since our variables are 0-1, we can accomplish this by maximizing the product of $X^i_{j, k} \cdot Y_{j, l}$ where $k$ and $l$ are adjacent. When both variables are assigned a value of 1, they contribute a single unit to the objective. Otherwise, they contribute nothing. \\
		However, this objective is \textit{not} linear. To make it linear, we introduce a set of helper variables:
			\begin{equation}
				Z^i_{j, k} \in \{0, 1\}: \text{represents the product }X^i_{j, k} \cdot Y_{j, 2k+1}
			\end{equation}
		with constraints:
			\begin{align}
				Z^i_{j, k} &\leq X^i_{j, k} \\
				Z^i_{j, k} &\leq Y_{j, 2k+1} \text{ for each } i
			\end{align}
		This gives us the third objective:
			\begin{equation}
				Obj_3\left(Z^i_{j, k}\right) \coloneqq \sum_i \sum_j \sum_k Z^i_{j, k}
			\end{equation}
	\end{enumerate}
	Putting all the components above together, we get a linear program defined as:
		\begin{equation}
			\begin{split}
			&\text{maximize } \frac{Obj_1 + Obj_2 + Obj_3}{3} \\
			&\text{subject to } (3)\text{-}(8)
			\end{split}
		\end{equation}
\end{document}
